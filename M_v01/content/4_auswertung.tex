\section{Auswertung}
\label{sec:Auswertung}

Für die Auswertung wird die \texttt{Python}-Bibliothek \texttt{numpy} \cite{numpy} benutzt. Die Fits entstehen mit \texttt{curve\_fit} aus \texttt{scipy.optimize} \cite{scipy}.
Die Fehlerrechnung wird mit \texttt{uncertainties} \cite{uncertainties} durchgeführt. Plots entstehen mit \texttt{matplotlib.pyplot} \cite{matplotlib}.

% Ergebnisse für die Auswertung
% Ich habe dir auch schon die nötigen Referenzen auf Gleichungen aus der Theorie geschrieben
% Du musst noch die Zahlen zu angemessenen Stellen runden

%%% Startsignale: 1496787
%%% Stoppsignale: 3144

%%%%%% Die Gleichung musst du einmal selber angeben, die konnte ich nicht gut in die Theorie packen
%%% gauss: f(x) = a * e^((x-x0)^2 / (2*sigma^2)
%%% a = 283.6143354+/-7.8672758 Einheit: Counts
%%% x0 = 0.6513704308+/-0.1750581592 Einheit: ns
%%% sigma = 5.066269628+/-0.196916370 Einheit: ns
%%% fwhm breite = 11.93015327+/-0.46370262 Einheit: ns

%%%%%% Referenz: \autoref{eq:linfit}
%%% linfit für kalibrierung: t(K) = m*K + n
%%% m =  0.021700+/-0.000015  Einheit: µs/Kanalnummer
%%% n =  0.1577+/-0.0035  Einheit: µs

% Ja, for some rason ist der Fehler hier in a und c RIESIG. idk warum
% außerdem: ich hab jetzt beim fit für lambda alle ersten vier werte ignoriert und die letzten paar hundert auch
% man kann aber auch argumentieren, den fit erst ab dem Peak zu machen (also so ab Kanal #24), weil der Anstieg an Counts in den Kanälen davor nicht erklärbar ist
% Hab ich jetzt aber halt nicht gemacht

%%%%%% Referenz: \autoref{eq:lambda}
%%% zerfallskurve = a*e^(-lambda*t - c)
%%% N_0 =  93.8+/-146477176.8, Einheit: Counts
%%% lambda =      0.575+/-     0.035 Einheit: 1/µs
%%% T = -1.815+/-2716074.511 Einheit: µs
%%% U = 0.7124202112+/-0.3260971519 Einheit: Counts
%%%  
%%% Mittlere Lebensdauer = 1/lambda = 1.739345761+/-0.104935602 Einheit: µs

%%%%%% Referenz: \autoref{eq:p} und \autoref{eq:U}
%%% Berechneter Untergrund: U = 0.2499813999936612 Einheit: Counts/Kanal

\subsection{Kalibrierung}

Der Delay wird wie in \autoref{sec:Kalibrierung} gemessen, die Messdaten sind in \autoref{fig:delay} dargestellt.

\begin{figure}[H]
    \centering
    \includegraphics[width=\textwidth]{plots/delay.pdf}
    \caption{Messung des Delays. Bei den blauen Messpunkten ist die Verzögerung in PM \enquote{A} eingestellt worden, bei den roten in PM \enquote{B}. Ebenfalls ist ein Gaußfit mit FWHM dargestellt.}
    \label{fig:delay}
\end{figure}
Es ist erkennbar, dass ein Delay in beide Richtungen zu einem Sinken der gemessenen Counts führt.

Der Gaußfit der Form
\begin{equation*}
    f(x) = a \cdot \exp\left((x-x_0)^2 / (2\sigma^2)\right),
\end{equation*}

$x$ ist der eingestellte Delay in Nanosekunden, hat die Parameter

\begin{align*}
    a &= \num{283.61(7.86)} \\
    x_0 &= \qty{0.651(175)}{\nano\second} \\
    \sigma &= \qty{5.07(20)}{\nano\second}
\end{align*}

mit einer FWHM-Breite von
\begin{equation*}
    \text{FWHM} = \qty{11.93(46)}{\nano\second}.
\end{equation*}

Die Kalibrierung für den MCA wird ebenfalls wie in \autoref{sec:Kalibrierung} beschrieben durchgeführt. Die Messdaten sind in \autoref{fig:kali} dargestellt.

\begin{figure}[H]
    \centering
    \includegraphics[width=\textwidth]{plots/kali.pdf}
    \caption{Kalibrierung des MCA. Dargestellt sind der eingestellte Impulsabstand gegen die zugeordnete Kanalnummer. Ebenfalls ist ein linearer Fit dargestellt.  }
    \label{fig:kali}
\end{figure}

Der lineare Fit nach \autoref{eq:linfit} hat die Parameter

\begin{align*}
    m &= \qty{0.021700(15)}{\micro\second\per\text{Kanal}} \\
    n &= \qty{0.1577(35)}{\micro\second}.
\end{align*}

Über diese Funktion können die Kanalnummern in Zeiten umgerechnet werden.


\subsection{Messung der Myonen}

Die Messung der Myonen wird wie in \autoref{sec:Messung} beschrieben durchgeführt. Die Kanäle werden wie zuvor beschrieben in Zeiten umgerechnet und die Counts der Kanäle gegen die Zeit aufgetragen, siehe \autoref{fig:zerfall}. Es werden \num{1496787} Startsignale und \num{3144} Stoppsignale gemessen.  

\begin{figure}[H]
    \centering
    \includegraphics[width=\textwidth]{plots/zerfall.pdf}
    \caption{Messung der Myonenlebensdauer. Dargestellt sind die gemessenen Counts gegen die Zeit. Ebenfalls ist ein exponentieller Fit nach \autoref{eq:lambda} dargestellt. Blau sind die dafür berücksichtigten Messpunkte, rot jene, die dafür ignoriert wurden.}
    \label{fig:zerfall}
\end{figure}

Der exponentielle Fit nach \autoref{eq:lambda} hat die Parameter
\begin{align*}
    N_0 &= \num{93.8} \pm \num{146477176.8} \\
    \lambda &= \qty{0.575(35)}{\per\micro\second} \\
    T &= \left(\num{-1.815} \pm \num{2716074.511}\right) \si{\micro\second} \\
    U &= \num{0.712(326)}.
\end{align*}

Für die mittlere Lebensdauer ergibt sich somit

\begin{equation*}
    \tau = \frac{1}{\lambda} = \qty{1.739(105)}{\micro\second}.
\end{equation*}

Neben dem Untergrund für den durchgeführten Fit kann dieser auch wie in \autoref{eq:p} und \ref{eq:U} beschrieben berechnet werden, es ergibt sich
\begin{equation*}
    U = \num{0.250} \, \text{Counts/Kanal}.
\end{equation*}

