\section{Auswertung}
\label{sec:Auswertung}

Für die Auswertung wird die \texttt{Python}-Bibliothek \texttt{numpy} \cite{numpy} benutzt. Die Fits entstehen mit \texttt{curve\_fit} aus \texttt{scipy.optimize} \cite{scipy}.
Die Fehlerrechnung wird mit \texttt{uncertainties} \cite{uncertainties} durchgeführt. Plots entstehen mit \texttt{matplotlib.pyplot} \cite{matplotlib}.

% Ergebnisse für die Auswertung
% Ich habe dir auch schon die nötigen Referenzen auf Gleichungen aus der Theorie geschrieben
% Du musst noch die Zahlen zu angemessenen Stellen runden

%%% Startsignale: 1496787
%%% Stoppsignale: 3144

%%%%%% Die Gleichung musst du einmal selber angeben, die konnte ich nicht gut in die Theorie packen
%%% gauss: f(x) = a * e^((x-x0)^2 / (2*sigma^2)
%%% a = 283.6143354+/-7.8672758 Einheit: Counts
%%% x0 = 0.6513704308+/-0.1750581592 Einheit: ns
%%% sigma = 5.066269628+/-0.196916370 Einheit: ns
%%% fwhm breite = 11.93015327+/-0.46370262 Einheit: ns

%%%%%% Referenz: \autoref{eq:linfit}
%%% linfit für kalibrierung: t(K) = m*K + n
%%% m =  0.021700+/-0.000015  Einheit: µs/Kanalnummer
%%% n =  0.1577+/-0.0035  Einheit: µs

% Ja, for some rason ist der Fehler hier in a und c RIESIG. idk warum
% außerdem: ich hab jetzt beim fit für lambda alle ersten vier werte ignoriert und die letzten paar hundert auch
% man kann aber auch argumentieren, den fit erst ab dem Peak zu machen (also so ab Kanal #24), weil der Anstieg an Counts in den Kanälen davor nicht erklärbar ist
% Hab ich jetzt aber halt nicht gemacht

%%%%%% Referenz: \autoref{eq:lambda}
%%% zerfallskurve = a*e^(-lambda*t - c)
%%% N_0 =  93.8+/-146477176.8, Einheit: Counts
%%% lambda =      0.575+/-     0.035 Einheit: 1/µs
%%% T = -1.815+/-2716074.511 Einheit: µs
%%% U = 0.7124202112+/-0.3260971519 Einheit: Counts
%%%  
%%% Mittlere Lebensdauer = 1/lambda = 1.739345761+/-0.104935602 Einheit: µs

%%%%%% Referenz: \autoref{eq:p} und \autoref{eq:U}
%%% Berechneter Untergrund: U = 0.2499813999936612 Einheit: Counts/Kanal