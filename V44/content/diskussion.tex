\section{Discussion}
\label{sec:Discussion}

First the relative deviations are calculated. In the second part of discussion possible reasons for these deviations are pointed out.

\subsection{Relative deviations}
In section \ref{sec:rockingscan} the geometry angle was measured as $\alpha_\text{g,mes}=\qty{1.296}{\degree}$ and theoretically calculated as $\alpha_\text{g,theo}=\qty{1.834}{\degree}$.
The relative deviation between these are 29.30\,\%.\par

The parratt algorithm gave the dispersions $\delta_\text{Si}=\num{7.672e-06}$ and $\delta_\text{Pol}=\num{1.124e-06}$. The literature values are 
$\delta_\text{Si,lit}=\num{7.6e-6}$ and $\delta_\text{Pol,lit}=\num{3.5e-6}$ \cite{V44:xrr_tolan}, which results in relativ deviations $\increment \delta_\text{Si}=0.95\,\%$ 
and $\increment \delta_\text{Pol}=67.88\,\%$.\par

Using the dispersion the critical angles have been calculated as $\alpha_\text{c,Si}=\qty{0.224}{\degree}$ and $\alpha_\text{c,Pol}=\qty{0.086}{\degree}$.
The literature values are $\alpha_\text{c,Si,lit}=\qty{0.223}{\degree}$ and $\alpha_\text{c,Pol,lit}=\qty{0.153}{\degree}$ \cite{V44:xrr_tolan}.
The relative deviations are $\increment \alpha_\text{c,Si}\qty{0.646}{\percent}$ and $\increment \alpha_\text{c,Pol}\qty{43.854}{\percent}$.\par

The layer thickness of the polystyrene layer has been calculated as $\qty{8.82(0.23)e-8}{\m}$ using the kiessig oscillations and as $\qty{8.551e-8}{\m}$ 
using the optimization of the parratt algorithm. The relative deviation between both is \qty{3.1(2.5)}{\percent}.

\subsection{Possible reasons for the deviations}
The relative deviation of the geometry angle is likely the result of the fact that not the whole plot was available and an approximation had to be done.\par
The relative deviations for $\delta_\text{Pol}$ cannot be explained but might be due to the way the optimization is performed.
The deviations of the critical angle are a direct result of those for the dispersion.
The relative deviation of \qty{3.1(2.5)}{\percent} for the determined layer thicknesses does not suggest any issues of the measurement.
Apart from that there are the usual errors when working with materials and intensities. The intensity is influenced by a background. 
Although the background was measured and subtracted it might have changed over time anyway. The sample might be damaged. Tutors have pointed out that there is a scratch on the sample, 
but it was not visible in the data.
