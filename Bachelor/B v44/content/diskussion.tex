\section{Diskussion}
\label{sec:Diskussion}

\subsection{Geometriewinkel}

Für den Geometriewinkel findet sich
\begin{align*}
    \alpha_\text{g, theo.} &= \qty{0.51(9)}{\degree} \\
    \alpha_\text{g, mess.} &= \qty{0.42(2)}{\degree} \\
    \Delta \alpha_\text{g} &= \num{17.6} \, \%.
\end{align*}

Eine Verbesserung wäre mit einer verlängerten Messzeit und feineren Messschritten möglich.

\subsection{Parratt-Algorithmus}

Der Parratt-Algorithmus liefert keinesfalls die besten Parameter, um die Messdaten zu beschreiben. Vergleiche mit Altprotokollen zeigen,
dass es möglich ist, die Kiessig-Oszillationen fast genau zu beschreiben, was hier aber nicht der Fall ist.

Für die Schichtdicke findet sich
\begin{align*}
    d_\text{Kiessig} &= \qty{8.83(25)e-8}{\meter} \\
    d_\text{Parratt} &= \qty{8.693e-8}{\meter} \\
    \Delta d &= \num{1.57} \, \%.
\end{align*}

Für die Dispersion ergibt sich mit den Literaturwerten
\begin{align*}
    \delta_\text{Si, Lit.} &= \num{7.6e-6} \\
    \delta_\text{Si, Parratt} &= \num{7.861e-6} \\
    \Delta \delta_\text{Si} &= \num{3.4} \, \%
\end{align*}

und

\begin{align*}
    \delta_\text{Poly., Lit.} &= \num{3.5e-6} \\
    \delta_\text{Poly., Parratt} &= \num{9.418e-7} \\
    \Delta \delta_\text{Poly} &= \num{73.0} \, \%.
\end{align*}

Für den kritischen Winkel $\alpha_\text{c}$ ergibt sich

\begin{align*}
    \theta_\text{c, Si, Lit} &= \qty{0.223}{\degree} \\
    \theta_\text{c, Si, Parratt} &= \qty{0.227}{\degree} \\
    \Delta \alpha_\text{c, Si} &= \num{1.7} \, \%.
\end{align*}

und 

\begin{align*}
    \theta_\text{c, Poly., Lit} &= \qty{0.153}{\degree} \\
    \theta_\text{c, Poly., Parratt} &= \qty{0.078}{\degree} \\
    \Delta \alpha_\text{c, Poly} &= \num{49.0} \, \%.
\end{align*}

Gründe für die Abweichungen sind unter anderem einmal, dass der Parratt-Algorithmus sichtlich verbesserungswürdig ist, wie bereits erklärt,
aber auch die Bedingungen der Probe. Kratzer und Flecken sind starke verunreinigungen und haben starken Einfluss auf die Oberflächenbeschaffenheit und dementsprechend
wirken sich Kratzer auch auf die Reflektivität aus. Diese Kratzer würden sich aber deutlich stärker in den Daten äußern, was hier nicht der Fall war, weshalb davon ausgegangen werden kann, 
dass keine signifikante Störstelle direkt bestrahlt wurde.

Nichtsdestotrotz ist die Abweichung der berechneten Schichtdicken über die Kiessig-Oszillationen und den Parratt-Algorithmus 
sehr gering mit nur $\num{1.57} \, \%$