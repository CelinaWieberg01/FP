\section*{Addendum}
\addcontentsline{toc}{section}{Addendum}
\begin{lstlisting}[language=python, caption=parrat\_algorithm]
#The parrameters are passed as arrays (shown at the end of the second code listing). The first index denotes the substrate layer and the last index the vacuum layer.
def parratt_reflectivity(angle, wavelength, layer_thickness, num_layers,delta,beta,sigma):
    X = np.zeros(num_layers+1,dtype=complex)
    r = np.zeros(num_layers+1, dtype=complex)
    k = 2 * np.pi / wavelength
    kz= np.zeros(num_layers+1,dtype=complex)
    n = np.zeros(num_layers+1,dtype=complex)
    n[0]=1-delta[0]-1j*beta[0]
    kz[0] = k*np.sqrt(n[0]**2-np.cos(np.deg2rad(angle))**2)

    for i in range(num_layers):
        n[i+1]=1-delta[i+1]-1j*beta[i+1]     
        kz[i+1] = k*np.sqrt(n[i+1]**2-np.cos(np.deg2rad(angle))**2)
        r[i] = ((kz[i+1]-kz[i])/(kz[i+1]+kz[i])) *  np.exp(-2*kz[i]*kz[i+1]*sigma[i]**2)
        X[i+1]=np.exp(-2j*kz[i+1]*layer_thickness[i])*((r[i] + X[i])/(1+r[i]*X[i] ))       
    return abs(X[-1])**2


def gen_parrat_curve(parrat_angle, wavelength, layer_thickness, delta, beta, sigma):
    parrat_reflectivity=np.empty(len(parrat_angle))
    for index,angle in enumerate(parrat_angle):
        parrat_reflectivity[index] = parratt_reflectivity(angle, wavelength, layer_thickness, len(layer_thickness),delta,beta,sigma)
    return parrat_reflectivity

\end{lstlisting}


\begin{lstlisting}[language=python, caption=code for the optimization]
#The distance metric called in the objective. Using a good distance metric is critical here!    
def distance(calced_dist,dist_comp):  
    result=np.sum(np.abs(np.log(1.0/(calced_dist+1e-6))-np.log(1.0/(dist_comp+1e-6))))*(1/len(calced_dist))  
    return result

#the objective function combines searchspaces and distance metric
def optuna_objective(trial,dist_comp,parrat_angle):
    layer_thickness = trial.suggest_float('layer_thickness', 8.5e-8,8.7e-8)
    delta1  = trial.suggest_float('delta1', 3e-6,8e-6 )
    delta2  = trial.suggest_float('delta2', 8e-7,3e-6 )
    beta1   = trial.suggest_float('beta1', 5e-9,1e-7  )
    beta2   = trial.suggest_float('beta2', 1e-9,5e-8  )
    sigma1  = trial.suggest_float('sigma1', 1e-11,1e-9)
    sigma2  = trial.suggest_float('sigma2', 1e-11,1e-9)
    return distance(gen_parrat_curve(parrat_angle,1.54e-10,[layer_thickness,0],[delta1,delta2,0],[beta1,beta2,0],[sigma1,sigma2]),dist_comp)

#This function is called to actually perform the parametersearch. 
#n_trials: determines how many parametercombinations should be tested
#dist_comp: data to compre the result of the parrat algorithm to (the measured data)
#parrat_angle: the corresponding angles to dist_comp
def parameter_search(n_trials,dist_comp,parrat_angle):
    optuna.logging.set_verbosity(optuna.logging.WARNING)
    study = optuna.create_study(study_name='parratt',sampler=(optuna.samplers.TPESampler(seed=1)))
    study.optimize(lambda trial: optuna_objective(trial,dist_comp,parrat_angle), n_trials=n_trials, catch=True,show_progress_bar=True)
    return study.best_params

parrat_angle_optuna = np.linspace(0, 2.5, len(geometry_corrected_relativity))
optuna_params=parameter_search(300,geometry_corrected_relativity[1:]/(5*max_intensity),parrat_angle_optuna[1:])
parratt_refl_optuna = gen_parrat_curve(parrat_angle, 1.54e-10,[optuna_params['layer_thickness'],0],[optuna_params['delta1'],optuna_params['delta2'],0],[optuna_params['beta1'],optuna_params['beta2'],0],[optuna_params['sigma1'],optuna_params['sigma2']])
\end{lstlisting}