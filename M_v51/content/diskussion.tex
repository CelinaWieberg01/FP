\section{Diskussion}
\label{sec:Diskussion}

Zuletzt werden die theoretischen Erkenntnisse und gemessenen Werte hinsichtlich ihrer Genauigkeit diskutiert.

\subsection{Inverting-Amplifier}

Die gemessenen Verstärkungsfaktoren von \num{88.0(7)}, \num{11.2(1)} und \num{106.0(4)} mit ihren Abweichungen um jeweils $\qty{12}{\percent}$, $\qty{12}{\percent}$ und $\qty{29}{\percent}$ von den Theoriewerten von $100$, $10$ und $150$
sind nicht groß. Die Abweichungen sind entweder auf Ungenauigkeiten der benutzten Widerstände als auch auf unbeachtete Impedanzen im Bredboard zurückzuführen.

Die Bandbreiten mit \num{1.1(5)e4}, \num{0.5(1.9)e5} und $\qty{8.0(3.3)e3}{\hertz}$ liegen zueinander entsprechend ihrer jeweiligen Verstärkung sinvoll.
Es wird beobachtet, dass größere Verstärkungen zu kleineren Bandbreiten führt.

\subsection{Integrator}
Das integrierende Verhalten eines Integrators ist klar demonstriert.
Des Weiteren ist die Abweichung der gemessenen zur errechneten Zeitkonstanten von \qty{0.00097(3)}{\second} zu \qty{0.001}{\second} mit \qty{2.6}{\percent} minimal.
\autoref{fig:int} lässt vermuten, dass ein Fit mit einer Potenzfunktion eventuell genauer sein könnte, allerdings entspricht das nicht den hier gestellten theoretischen Erwartungen.

\subsection{Differentiator}
Das differenzierende Verhalten eines Differentiators ist klar demonstriert.
Die Abweichung der gemessenen zur errechneten Zeitkonstanten von \qty{0.0022}{\second} zu \qty{0.00185(1)}{\second} ist mit \qty{16}{\percent} größer als beim Integrator.
Der lineare Fit in \autoref{fig:dif} passt sehr gut auf die Messdaten.

\subsection{Schmitt-Trigger}
Die Abweichung zwischen der theoretischen Schwellspannung von \qty{2.81(1)}{\volt} und der gemessenen Spannung von \qty{2.980(1)}{\volt} beträgt \qty{6}{\percent}.
Es wäre besser gewesen, wäre eine Ausgangssättigungsspannung vorher gemessen oder berechnet worden, um einen unabhängigen Verleich zwischen Theorie und Experiment für die Schwellspannung zu erhalten. So konnte die Sättigungsspannung
erst aus dem Bild der Messung entnommen werden,
was die Vergleichbarkeit zwischen einem Theoriewert und des experimentell bestimmten Wertes für die Schwellspannung redundant macht.


\subsection{Generatoren}
Der Frequenzunterschied von \qty{34}{\percent} zwischen der theoretischen Frequenz von \qty{2500}{\hertz} und der ermittelten Frequenz von \SI{1.65(10)e3}{\hertz}
ist etwas groß. Das ist damit zu begründen, dass das entstehende Rechtecksignal des Schmitt-Triggers Unstetigkeiten schräg ansteigende Flanken besitzt, wie in \autoref{fig:schm} erkennbar ist.
Dies führt dazu, dass das Dreiecksignal, also das Integral des Outputs, an den Spitzen Unstetigkeiten aufweist, was die automatische Auslese vom Oszilloskop möglicherweise stört.

Es konnte kein herkömmlicher Fit einer Lösung auf die Messwerte für den variablen Generator durchgeführt werden. 
Die Abweichung in $T$ von \qty{0.00628}{\second} mit \qty{0.00536(16)}{\second} bzw. \qty{0.00542(25)}{\second} betragen $\qty{15}{\percent}$ bzw. $\qty{14}{\percent}$.
Die Abweichungen in $\tau$ vom Theoriewert von \qty{0.02}{\second} für die maximale Dämpfung mit \qty{0.081(25)}{\second} lautet \qty{310}{\percent}.
Die Abeweichung für minimale Dämpfung mit \qty{0.027(6)}{\second} lautet \qty{38}{\percent}.

Die Bestimmung des Dämpfungsfaktors ist nicht quantitativ, sondern nur qualitativ. Das Potentiometer wurde in beiden Messungen bis zum Anschlag hoch- und runtergeregelt.
Die genaue Größe des Faktors wurde somit nicht bestimmt, was die große Abweichung für eine Abklingdauer gemäß \autoref{eq:tau} erklärt.