\section{Diskussion}
\label{sec:Diskussion}

Zuletzt werden die theoretischen Erkenntnisse und gemessenen Werte hinsichtlich ihrer Genauigkeit diskutiert.

\subsection{Inverting-Amplifier}

Die gemessenen Verstärkungsfaktoren mit den dazugehörigen Bandbreiten betragen \num{88.0(7)} mit \qty{1.1(5)e4}{\hertz}, \num{11.2(1)} mit \qty{0.5(1.9)e5}{\hertz} und \num{106.0(4)} mit \qty{8.0(3.3)e3}{\hertz}.
Die Abweichungen der Verstärkungsfaktoren um jeweils $\qty{12}{\percent}$, $\qty{12}{\percent}$ und $\qty{29}{\percent}$ von den Theoriewerten von $100$, $10$ und $150$
sind nicht groß. Die Abweichungen sind entweder auf Ungenauigkeiten der benutzten Widerstände oder auf unbeachtete Impedanzen im Bredboard zurückzuführen.
Die Bandbreiten nehmen mit größerer Verstärkung ab und kleinerer Verstärkung zu.

\subsection{Integrator}
Das integrierende Verhalten eines Integrators ist klar demonstriert.
Des Weiteren ist die Abweichung der gemessenen zur errechneten Zeitkonstanten von \qty{0.00097(3)}{\second} zu \qty{0.001}{\second} mit \qty{2.6}{\percent} minimal.
\autoref{fig:int} lässt vermuten, dass ein Fit mit einer Potenzfunktion eventuell genauer sein könnte, allerdings entspricht das nicht den hier gestellten theoretischen Erwartungen.
Da der Kondensator parallel zum OP-Amp geschaltet ist, kann es zu Abweichungen in der Verstärkung kommen, was in \autoref{fig:int} bei hohen inversen Kreisfrequenzen erkennbar ist.

\subsection{Differentiator}
Das differenzierende Verhalten eines Differentiators ist klar demonstriert.
Die Abweichung der gemessenen zur errechneten Zeitkonstanten von \qty{0.0022}{\second} zu \qty{0.00185(1)}{\second} ist mit \qty{16}{\percent} größer als beim Integrator.
Der lineare Fit in \autoref{fig:dif} passt allerdings sehr gut zu den Messdaten. Integratoren glätten eingehende Signale, also auch Rauschen, während Differentiatoren
stärker auf Rauschsignale reagieren \cite{tiefpass}. Das Hochpassverhalten lässt viele Rauschquellen im höheren Frequenzbereich durch, sodass die hier gemessenen Signale größere Fehler beinhalten.
Auch die Schaltung beeinflusst hier die Signalform, da der in Reihe geschaltete Kondensator bei Unregelmäßigkeiten diese hier verändert.

\subsection{Schmitt-Trigger}
Die Abweichung zwischen der theoretischen Schwellspannung von \qty{2.81(1)}{\volt} und der gemessenen Spannung von \qty{2.980(1)}{\volt} beträgt \qty{6}{\percent}, ist also sehr klein.


\subsection{Generatoren}
Der Frequenzunterschied von \qty{34}{\percent} zwischen der theoretischen Frequenz von \qty{2500}{\hertz} und der ermittelten Frequenz von \SI{1.65(10)e3}{\hertz}
ist verglichen mit anderen berechneten Abweichungen groß. Die Frequenz ist, wie in \autoref{eq:frequenz} zu sehen ist, direkt von den Größen der verwendeten Bauteile abhängig, die außerdem eine hohe Temperaturabhängigkeit
aufweisen. 

Es konnte kein herkömmlicher Fit einer Lösung auf die Messwerte für den variablen Generator durchgeführt werden. 
Die theoretische, berechnete Periodendauer $T$ beträgt \qty{0.00628}{\second} Die gemesesnen Periodendauern betragen \qty{0.00536(16)}{\second} und \qty{0.00542(25)}{\second}. Die entsprechenden Abweichungen betragen $\qty{15}{\percent}$ bzw. $\qty{14}{\percent}$.
Die theoretische, berechnete Abklingzeit $\tau$ beträgt \qty{0.02}{\second}. Die gemessene Abklingzeit bei maximaler Dämpfung beträgt \qty{0.081(25)}{\second}. Die Abweichung beträgt \qty{310}{\percent}.
Die gemessene Abklingzeit bei minimaler Dämpfung beträgt \qty{0.027(6)}{\second}, die Abweichung beträgt \qty{38}{\percent}.

Die Bestimmung des Dämpfungsfaktors ist nicht quantitativ, sondern nur qualitativ. Das Potentiometer wurde in beiden Messungen bis zum Anschlag hoch- und runtergeregelt.
Das bedingt die großen Abweichungen der Theoriewerte. Über die berechneten Abklingzeiten $\tau$ können Werte für die maximale und minimale Dämpfung berechnet werden, diese betragen 
$\left|\eta_{+1}\right| = \num{0.25(8)}$ und $\left|\eta_{-1}\right| = \num{0.73(16)}$. Die Richtigkeit dieser Werte ließe sich in Messungen bei mehr und verschiedenen als nur den extremalen Dämpfungen genauer bestimmen.
