\section{Diskussion}
\label{sec:Diskussion}

Zuletzt werden die theoretischen Erkenntnisse und gemessenen Werte hinsichtlich ihrer Genauigkeit diskutiert.

\subsection{Inverting-Amplifier}

Die gemessenen Verstärkungsfaktoren von \num{88.0(7)}, \num{11.2(1)} und \num{106.0(4)} mit ihren Abweichungen um jeweils $12$, $12$ und $29 \%$
sind angemessen. Abweichungen sind wahrscheinlich auf Ungenauigkeiten der benutzten Widerstände zurückzuführen. Diese hätten vor Benutzen vermessen werden können, um genauere Werte zu erhalten.

Die Bandbreiten mit \num{1.1(5)e4}, \num{0.5(1.9)e5} und $\qty{8.0(3.3)e3}{\hertz}$ liegen zueinander in Anbetracht ihrer entsprechenden Verstärkung gut,
eine größere Verstärkung führt zu einer kleineren Bandbreite und umgekehrt, was hier beobachtet werden kann.

\subsection{Integrator}
Das integrierende Verhalten eines Integrators ist klar demonstriert.
Des Weiteren ist die Abweichung der gemessenen zur errechneten Zeitkonstanten von \qty{0.00097(3)}{\second} zu \qty{0.001}{\second} mit \qty{2.6}{\percent} minimal.
\autoref{fig:int} lässt vermuten, dass ein Fit mit einer Potenzfunktion eventuell genauer sein könnte, allerdings entspricht das nicht den hier gestellten theoretischen Erwartungen.
Es ist zu erwarten, dass die benutzten Bauteile korrekte Labels bzgl. ihrer Größen besitzen.

\subsection{Differentiator}
Das differezierende Verhalten eines Differentiators ist klar demonstriert.
Die Abweichung der gemssenen zur errechneten Zeitkonstanten von \qty{0.0022}{\second} zu \qty{0.00185(1)}{\second} mit \qty{16}{\percent} ist deutlich größer als beim Integrator, aber im verkraftbaren Bereich.
Der lineare Fit in \autoref{fig:dif} passt sehr gut auf die Daten.
Dies lässt darauf deuten, dass die benutzte Kapazität ein falsches Label trägt, da derselbe Widerstand wie beim Integrator verwendet wurde.

\subsection{Schmitt-Trigger}
Der Unterschied von \qty{6}{\percent} zwischen der theoretischen Schwellspannung von \qty{2.81(1)}{\volt} und der gemessenen Spannung von \qty{2.980(1)}{\volt} ist akzeptabel.
Es wäre besser gewesen, wäre die Ausgangssättigungsspannung vorher berechnet worden. So musste die Messung erst gemacht werden, um einen Theoriewert zu erhalten.
Es konnte ebenfalls die Funktionalität des Schmitt-Triggers demonstriert werden.


\subsection{Generatoren}
Der Frequenzunterschied von \qty{34}{\percent} zwischen der theoretischen Frequenz von \qty{34}{\percent} und der ermittelten Frequenz von \SI{1.65(10)e3}{\hertz}
ist etwas groß. Das ist wahrscheinlich darauf zu schieben, dass das entstehende Rechtecksignal des Schmitt-Triggers nicht ganz sauber ist, wie in \autoref{fig:schm} erkennbar:
da dieser deutlich schräg ansteigende Flanken besitzt, ist das Dreiecksignal, also das Integral, an den Spitzen sehr unsauber, was die Frequenzauslese erschwert.

Obwohl kein herkömmlicher Fit einer Lösung auf die Messwerte für den variablen Generator möglich war, sind die gefunden Werte teilweise im Rahmen des Verkraftbaren.
Die Abweichung in $T$ mit $\qty{15}{\percent}$ bzw. $\qty{14}{\percent}$ ist unerwartet klein, es war zu erwarten, dass aufgrund der ungeauen Lage der peaks diese viel größer sein würde.

Die Abweichung in $\tau$, die wiederum mit einem anderen Fit entstand, ist für die maximale Dämpfung mit \qty{310}{\percent} so hoch, dass dieser vermuten lässt, dass etwas in der Durchführung falsch lief,
während die Abeweichung für minimale Dämpfung mit \qty{38}{\percent} wieder in einem akzeptablen Bereich liegt.

