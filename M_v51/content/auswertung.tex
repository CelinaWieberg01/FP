\section{Auswertung}
\label{sec:Auswertung}
\subsection{Invertierender Verstärker}
Es wurden drei Messreihen eines invertierenden Operationsverstärkers mit unterschiedlichen Widerstandskombinationen aufgenommen:
\begin{itemize}
  \item Messreihe 1: \(R_1 = 1\ \text{k}\Omega\), \(R_2 = 100\ \text{k}\Omega\).
  \item Messreihe 2: \(R_1 = 1\ \text{k}\Omega\), \(R_2 = 10\ \text{k}\Omega\).
  \item Messreihe 3: \(R_1 = 1\ \text{k}\Omega\), \(R_2 = 150\ \text{k}\Omega\).
\end{itemize}

Die (betragsmäßige) ideale Spannungsverstärkung eines invertierenden Verstärkers ist
\begin{equation}
  V_{\mathrm{theo}}=\left|\!-\frac{R_2}{R_1}\right|=\frac{R_2}{R_1},
\end{equation}
woraus sich die theoretischen Werte
\begin{equation}
  V_{1,\mathrm{theo}}=100,\quad V_{2,\mathrm{theo}}=10,\quad V_{3,\mathrm{theo}}=150
\end{equation}
ergeben.

Zur Analyse des Frequenzverhaltens wurde die Ausgangsamplitude \(U_{\mathrm{a}}(\nu)\) gemessen und die Verstärkung als
\begin{equation}
  V(\nu)=\frac{U_{\mathrm{a}}(\nu)}{U_{\mathrm{ein}}}
\end{equation}
berechnet. Außerhalb des Plateaus beschreibt ein Potenzgesetz das Abfallen der Verstärkung:
\begin{equation}\label{eq:powerlaw}
  V(\nu)=a\,\nu^{\,b},
\end{equation}
wobei \(a\) und \(b\) durch nichtlineare Regression bestimmt werden. Die Plateauverstärkung \(U_{\mathrm{konst}}\) wurde als Mittelwert der Messwerte im Plateaubereich (hier: \(\nu<1000\ \text{Hz}\)) bestimmt.

Die Grenzfrequenz \(\nu_{\mathrm{Grenze}}\) als Schnittpunkt der Potenzkurve mit \(U_{\mathrm{konst}}\) ergibt sich aus
\begin{equation}\label{eq:cutoff}
  U_{\mathrm{konst}} = a\,\nu_{\mathrm{Grenze}}^{\,b}
  \quad\Rightarrow\quad
  \nu_{\mathrm{Grenze}} = \left(\frac{U_{\mathrm{konst}}}{a}\right)^{1/b}.
\end{equation}
Die Bandbreite wird als Differenz der rechten und linken Grenzfrequenz definiert:
\begin{equation}\label{eq:bandwidth}
  \mathrm{Bandbreite}=\nu_{\mathrm{Grenze,right}}-\nu_{\mathrm{Grenze,left}}.
\end{equation}

\subsubsection*{Messreihe 1 (R\_2 = 100 kΩ, \(U_{\mathrm{ein}} = 2.5\ \text{V}\))}
Aus den Fits der Flanken ergaben sich folgende Parameter:
\begin{align}
  \text{linke Flanke:}\quad & a_{1}=4.0265,\quad b_{1}=-0.01198,\\
  \text{rechte Flanke:}\quad & a_{2}=229.3526,\quad b_{2}=-0.48834,
\end{align}
und als Plateauwert \(U_{\mathrm{konst}} = 3.81\).

Eingesetzt in Gleichung~\eqref{eq:cutoff} ergibt sich
\begin{align}
  \nu_{\mathrm{left}} &\approx 1.0088\times 10^{2}\ \text{Hz},\\
  \nu_{\mathrm{right}} &\approx 4.4071\times 10^{3}\ \text{Hz},
\end{align}
woraus sich die Bandbreite
\begin{equation}
  \mathrm{Bandbreite}\approx 4.3062\times 10^{3}\ \text{Hz}
\end{equation}
ergibt. Die sehr flache linke Flanke (kleiner Exponent \(b\) nahe Null) zeigt, dass der Verstärker bei tiefen Frequenzen nahezu konstant verstärkt; die negative Steigung der rechten Flanke entspricht dem erwarteten Abfall bei hohen Frequenzen.

\begin{figure}[htbp]
  \centering
  \includegraphics[width=0.8\textwidth]{Plots/invamp1.png}
  \caption{Messreihe 1: Verstärkung \(V=U_{\mathrm{a}}/U_{\mathrm{ein}}\) gegen Frequenz \(\nu\) (log‑log). Rote Linie: Potenzfit \(a\nu^{b}\). Orange gestrichelt: ln–ln Ausgleichsgerade. Grüne Linie: \(U_{\mathrm{konst}}\). Graue Linien: berechnete Grenzfrequenzen.}
  \label{fig:inv1}
\end{figure}

\subsubsection*{Messreihe 2 (R\_2 = 10 kΩ, \(U_{\mathrm{ein}} = 90\ \text{mV}\))}
Die Fits lieferten:
\begin{align}
  \text{linke Flanke:}\quad & a_{3}=0.4799,\quad b_{3}\approx 2.12\times 10^{-9},\\
  \text{rechte Flanke:}\quad & a_{4}=0.54513,\quad b_{4}=-0.03477,
\end{align}
mit Plateau \(U_{\mathrm{konst}}=0.48\).

Einsetzen in \eqref{eq:cutoff} ergibt formal
\begin{align}
  \nu_{\mathrm{left}} &\approx 1.2336\times 10^{2}\ \text{Hz},\\
  \nu_{\mathrm{right}} &\approx 3.8839\times 10^{1}\ \text{Hz},
\end{align}
wodurch die formale Bandbreite negativ wird:
\begin{equation}
  \mathrm{Bandbreite}\approx -8.4520\times 10^{1}\ \text{Hz}.
\end{equation}
Diese negative Bandbreite ist physikalisch nicht sinnvoll und weist auf Probleme bei der Fitwahl bzw. der Plateaubestimmung hin (z.\,B. zu schlecht bestimmter Exponent \(b_3\), vertauschte Fitbereiche oder Ausreißer). Vor einer finalen Interpretation sollten die Fitbereiche überprüft und der Fit ggf. neu gewählt werden.

\begin{figure}[htbp]
  \centering
  \includegraphics[width=0.8\textwidth]{Plots/invamp2.png}
  \caption{Messreihe 2: Verstärkung gegen Frequenz (Darstellung wie in Abb.~\ref{fig:inv1}). Auffällig ist der nahezu verschwindende Exponent der linken Flanke, was zu nichtrobusten Schnittpunkten führt.}
  \label{fig:inv2}
\end{figure}

\subsubsection*{Messreihe 3 (R\_2 = 150 kΩ, \(U_{\mathrm{ein}} = 500\ \text{mV}\))}
Die Fitparameter lauten:
\begin{align}
  \text{linke Flanke:}\quad & a_{5}=1.8287,\quad b_{5}=-0.01405,\\
  \text{rechte Flanke:}\quad & a_{6}=177.2094,\quad b_{6}=-0.56579,
\end{align}
mit Plateau \(U_{\mathrm{konst}}=1.709\). Daraus folgen
\begin{align}
  \nu_{\mathrm{left}} &\approx 1.2365\times 10^{2}\ \text{Hz},\\
  \nu_{\mathrm{right}} &\approx 3.6536\times 10^{3}\ \text{Hz},
\end{align}
und damit
\begin{equation}
  \mathrm{Bandbreite}\approx 3.5299\times 10^{3}\ \text{Hz}.
\end{equation}

\begin{figure}[htbp]
  \centering
  \includegraphics[width=0.8\textwidth]{Plots/invamp3.png}
  \caption{Messreihe 3: Verstärkung gegen Frequenz (Darstellung wie in Abb.~\ref{fig:inv1}).}
  \label{fig:inv3}
\end{figure}


\clearpage
\subsection{Integrator}
Für den Integrator wurden die Ausgangsamplitude \(U_A\) als Funktion der Frequenz gemessen. Verwendete Bauteile: \(C = 100\ \text{nF}\) und \(R = 10\ \text{k}\Omega\). Die Daten wurden mit Gleichung~\eqref{eq:powerlaw} gefittet; die erhaltenen Parameter lauten
\begin{equation}
  a_{\mathrm{int}} = 63.136 \pm 8.55,\qquad b_{\mathrm{int}} = -0.722 \pm 0.050.
\end{equation}
Gemäß der im Versuch verwendeten Herleitung wurde aus \(a_{\mathrm{int}}\) die konstante Größe \(C_{\mathrm{konst,int}}\) berechnet:
\begin{equation}
  C_{\mathrm{konst,int}} = 159.155.
\end{equation}
Die Messdaten und die Fitkurve sind in Abbildung~\ref{fig:integrator} dargestellt.

\begin{figure}[htbp]
  \centering
  \includegraphics[width=0.8\textwidth]{Plots/integrator.png}
  \caption{Integrator: Amplitude \(V=U_A/U_{\mathrm{ein}}\) gegen Frequenz; rote Linie: Potenzfit, orange gestrichelt: ln–ln Ausgleichsgerade.}
  \label{fig:integrator}
\end{figure}

Bei Rechteck‑ bzw. Dreieckseingängen zeigt der Integrator das erwartete Verhalten.

\begin{figure}[htbp]
  \centering
  \begin{subfigure}[t]{0.32\textwidth}
    \centering
    \includegraphics[width=\linewidth]{data/bilder/integrator_rechteck.png}
    \caption{Integrator bei Rechteck‑Eingang.}
    \label{fig:int_rect}
  \end{subfigure}\hfill
  \begin{subfigure}[t]{0.32\textwidth}
    \centering
    \includegraphics[width=\linewidth]{data/bilder/integrator_dreieck.png}
    \caption{Integrator bei Dreieck‑Eingang.}
    \label{fig:int_tri}
  \end{subfigure}\hfill
  \begin{subfigure}[t]{0.32\textwidth}
    \centering
    \includegraphics[width=\linewidth]{data/bilder/integrator_sinus.png}
    \caption{Integrator bei Sinus‑Eingang.}
    \label{fig:int_sine}
  \end{subfigure}
  \caption{Zeitbereichsantworten des Integrators für drei Eingangssignale (grün: Eingang, gelb: Ausgang).}
  \label{fig:int_time}
\end{figure}

\smallskip

Kurzbeschreibung der Beobachtungen:
\begin{itemize}
  \item \textbf{Rechteck (Abb.~\ref{fig:int_rect}):} Der Integrator wandelt ein Rechtecksignal in eine annähernd dreieckförmige Ausgangsspannung um, da die Integration einer Sprungfolge zu linearen Anstiegen und Abfällen führt.
  \item \textbf{Dreieck (Abb.~\ref{fig:int_tri}):} Bei einem Dreieckseingang ergibt sich am Ausgang ein stückweise quadratischer Verlauf (ungefähr parabolische Form), da die Integration einer konstanten Steigung eine linear wachsende Amplitude liefert, deren Ableitung wiederum linearer Verlauf ist.
  \item \textbf{Sinus (Abb.~\ref{fig:int_sine}):} Bei sinusförmigem Eingang erscheint am Ausgang ebenfalls ein Sinus, jedoch phasenverschoben um etwa \(-90^\circ\) und mit amplitudenabhängiger Skalierung entsprechend dem Frequenzverhalten des Integrators.
\end{itemize}

\subsection{Differentiator}
Für den Differentiator wurden die Bauteile \(C = 2\ \text{nF}\) und \(R = 100\ \text{k}\Omega\) verwendet. Die Fitparameter sind
\begin{equation}
  a_{\mathrm{diff}} = 0.017 \pm 0.003,\qquad b_{\mathrm{diff}} = 0.958 \pm 0.035.
\end{equation}
Daraus wurde die konstante Größe berechnet:
\begin{equation}
  C_{\mathrm{konst,diff}} = 0.014.
\end{equation}
Die Messdaten und der Fit sind in Abbildung~\ref{fig:differentiator} dargestellt. Der Exponent \(b_{\mathrm{diff}}\) nahe \(1\) ist konsistent mit differentiellen Verhalten (Amplitude proportional zu \(\nu\)) im betrachteten Frequenzbereich.

\begin{figure}[htbp]
  \centering
  \includegraphics[width=0.8\textwidth]{Plots/differentiator.png}
  \caption{Differentiator: Amplitude \(V=U_A/U_{\mathrm{ein}}\) gegen Frequenz; Darstellung wie in Abb.~\ref{fig:integrator}.}
  \label{fig:differentiator}
\end{figure}

Nach Auswertung der Amplitudenfrequenzgänge wurden die zeitlichen Ausgangssignale des Differentiators für drei standardisierte
Eingangssignale aufgenommen: Rechteck, Dreieck und Sinus. Abbildung~\ref{fig:diff_time} zeigt die gemessenen Verläufe, das Eingangssignal ist grün, das Ausgangssignal gelb.

\begin{figure}[htbp]
  \centering
  \begin{subfigure}[t]{0.32\textwidth}
    \centering
    \includegraphics[width=\linewidth]{data/bilder/differentiator_rechteck.png}
    \caption{Differentiator bei Rechteck‑Eingang.}
    \label{fig:diff_rect}
  \end{subfigure}\hfill
  \begin{subfigure}[t]{0.32\textwidth}
    \centering
    \includegraphics[width=\linewidth]{data/bilder/differentiator_dreieck.png}
    \caption{Differentiator bei Dreieck‑Eingang.}
    \label{fig:diff_tri}
  \end{subfigure}\hfill
  \begin{subfigure}[t]{0.32\textwidth}
    \centering
    \includegraphics[width=\linewidth]{data/bilder/differentiator_sinus.png}
    \caption{Differentiator bei Sinus‑Eingang.}
    \label{fig:diff_sine}
  \end{subfigure}
  \caption{Zeitbereichsantworten des Differentiators für drei Eingangssignale.}
  \label{fig:diff_time}
\end{figure}

\smallskip

\begin{itemize}
  \item \textbf{Rechteck (Abb.~\ref{fig:diff_rect}):} Der Differentiator erzeugt an den Flanken kurze, starke Impulse (positive bei ansteigender Flanke, negativ bei fallender Flanke). Das ist typisch, da die Ableitung eines Rechtecks Nadel‑ bzw. Impuls‑artige Anstiege ergibt.
  \item \textbf{Dreieck (Abb.~\ref{fig:diff_tri}):} Bei einem Dreieckssignal ergibt sich am Ausgang eine stückweise konstante Amplitude, da die Ableitung eines Dreiecks (konstante Steigung) nahezu rechteckartige Abschnitte liefert.
  \item \textbf{Sinus (Abb.~\ref{fig:diff_sine}):} Für einen sinusförmigen Eingang ergibt der Differentiator ebenfalls eine sinusförmige Ausgangsspannung, phasenverschoben um etwa 90° und amplitudenproportional zur Frequenz; dies ist in der Messkurve sichtbar als Sinus mit erkennbarer Phasenverschiebung und ggf. veränderter Amplitude.
\end{itemize}

\subsection{Schmitt-Trigger
Aus den in der Durchführung verwendeten Widerstandskonfigurationen ergibt sich ein theoretischer Schwellwert von
\begin{equation}
  U_{\pm}=\pm 2.98\ \mathrm{V}.
\end{equation}
Experimentell wurde der Schwellwert durch Variation der Eingangsspannung ermittelt, wobei bei einer gemessenen Eingangsspannung
\begin{equation}
  U_{\text{Schwell}} = 2.985\pm0.001 \mathrm{V}
\end{equation}
ein stabiler Rechteckbetrieb mit der erwarteten Amplitude auftrat. Die Messkurve am ermittelten Schwellwert ist in Abbildung~\ref{fig:schwell} dargestellt, sie zeigt den Übergang in das stabile Rechteckverhalten.

\begin{figure}[htbp]
  \centering
  \includegraphics[width=0.8\textwidth]{data/bilder/scope_6.png}
  \caption{Messung zur Bestimmung des Schwellwertes. Darstellung der Eingangsspannung und des stabilen Rechtecksignals bei \(U_{\text{Schwell}}=2.985\mathrm{V}\).}
  \label{fig:schwell}
\end{figure}}

\subsection{Generator}
Im Rahmen der Messung wurde das Oszilloskopbild aufgenommen, das den angelegten sinusförmigen Generatorausgang zeigt.
Aus dem Bild können die Messgrößen direkt abgelesen werden. Die Signalfrequenz beträgt \(\nu = 1.6465\ \text{Hz}\) und der gemessene Spannungswert liegt bei \(U = 5.4\ \text{V}\).
Die zugehörige Aufnahme ist in Abbildung~\ref{fig:generator_signal} eingefügt. Sie dient sowohl der Dokumentation der verwendeten Anregung 
als auch der Kontrolle, dass die eingestellten Parameter während der Messung stabil waren.

\begin{figure}[htbp]
  \centering
  \includegraphics[width=0.8\textwidth]{M_v51/data/bilder/scope_7.png}
  \caption{Oszilloskopaufnahme des an den Versuch angelegten sinusförmigen Signals.}
  \label{fig:generator_signal}
\end{figure}

\subsection{Generator mit variabler Amplitude}










