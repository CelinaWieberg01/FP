\section{Auswertung}
\label{sec:Auswertung}
Für die Auswertung wird die \texttt{Python}-Bibliothek \texttt{numpy} \cite{numpy} benutzt. Die Fits entstehen mit \texttt{curve\_fit} aus \texttt{scipy.optimize} \cite{scipy}.
Die Fehlerrechnung wird mit \texttt{uncertainties} \cite{uncertainties} durchgeführt. Plots entstehen mit \texttt{matplotlib.pyplot} \cite{matplotlib}.

\subsection{Inverting Amplifier}
Zur Überprüfung der Funktionalität eines invertierenden Operationsverstärkers wurden drei Messreihen mit unterschiedlichen Widerstandskombinationen
durchgeführt. Dabei wurden jeweils der Eingangs- und der Rückkopplungswiderstand variiert:

\begin{itemize}
  \item Messung 1: $R_1 = \SI{1}{\kilo\ohm}$, $R_2 = \SI{100}{\kilo\ohm}$
  \item Messung 2: $R_1 = \SI{1}{\kilo\ohm}$, $R_2 = \SI{10}{\kilo\ohm}$
  \item Messung 3: $R_1 = \SI{1}{\kilo\ohm}$, $R_2 = \SI{150}{\kilo\ohm}$ 
\end{itemize}

Die theoretische Spannungsverstärkung eines invertierenden Verstärkers ergibt sich aus der Formel


\[
V_\text{theo} = \left| -\frac{R_2}{R_1} \right| = \frac{R_2}{R_1}
\]


Das Minuszeichen steht für die Phasenumkehr des Ausgangssignals. Die berechneten Verstärkungen lauten:


\[
V_{1,\text{theo}} = 100,\quad V_{2,\text{theo}} = 10,\quad V_{3,\text{theo}} = 150
\]
Zur Analyse der Frequenzabhängigkeit der Verstärkung wurde die Ausgangsspannung $U_\text{a}$ in Abhängigkeit von der Frequenz $\nu$ gemessen 
und die Verstärkung als Quotient $V = U_\text{a} / U_\text{ein}$ berechnet. Anschließend wurde die Verstärkung gegen die Frequenz in einem 
doppelt-logarithmischen Diagramm dargestellt.
Im Bereich außerhalb des Verstärkungsplateaus zeigt die Verstärkung ein charakteristisches Abfallverhalten, das durch eine Potenzfunktion beschrieben werden kann:


\[
V(\nu) = a \cdot \nu^b
\]


Die Parameter $a$ und $b$ werden jeweils durch eine Ausgleichsrechnung bestimmt.
Die Grenzfrequenzen ergeben sich aus den Schnittpunkten der
Fit-Funktion mit der konstanten Verstärkung. Durch Umstellen der Gleichung ergibt sich:


\[
U_\text{konst} = a \cdot \nu_\text{Grenze}^b
\quad\Rightarrow\quad
\nu_\text{Grenze} = \left( \frac{U_\text{konst}}{a} \right)^{1/b}
\]

Für jede Messreihe werden zwei Grenzfrequenzen bestimmt, eine am unteren und eine am oberen Ende des Plateaus. 
Die Differenz dieser beiden Werte ergibt die Bandbreite der jeweiligen Verstärkerkonfiguration:

\[
\text{Bandbreite} = \nu_\text{Grenze_{rechts}} - \nu_\text{Grenze_{links}}
\]

Die Fit-Parameter $a$ und $b$ sowie die konstante Verstärkung $U_\text{konst}$ werden aus den jeweiligen Plots abgelesen und zur Berechnung der Grenzfrequenzen verwendet.

Für die erste Messreihe ergaben sich die Fit‑Parameter

\[
a_{1}=229.352605,\qquad b_{1}=-0.011981\quad
\]

und

\[
a_{2}=229.35260,\qquad b_{2}=-0.488338\quad
\]


Die gemessene Plateauverstärkung beträgt \(U_{\text{konst}}=3.81\). Aus diesen Werten folgen die Grenzfrequenzen
\(\nu_{\text{links}}\approx 100.881520\mathrm{Hz}\) und \(\nu_{\text{rechts}}\approx 4407.099025,\mathrm{Hz}\), 
somit ergibt sich eine Bandbreite von etwa 4306.217505\mathrm{Hz}\.
Diese Werte sind plausibel, die sehr flache linke Flanke (kleiner Exponent \(b\) nahe Null) zeigt, dass der Verstärker bei tiefen Frequenzen praktisch 
konstant verstärkt, während das deutlich negative \(b\) der rechten Flanke den erwarteten Abfall der Verstärkung mit steigender Frequenz widerspiegelt.

\paragraph{Plot}
\begin{figure}[htbp]
  \centering
  \includegraphics[width=0.8\textwidth]{Plots/invamp1.png}
  \caption{Plot zur ersten Messreihe. Die verwendeten Wiederstände betragen $R_1 = \SI{1}{\kilo\ohm}$, $R_2 = \SI{100}{\kilo\ohm}$}
  \label{fig:inv1}
\end{figure}


Die zweite Messreihe liefert die Fit‑Parameter


\[
a_{3}=0.4799,\qquad b_{3}\approx 2.12446\times10^{-9}\quad
\]

und

\[
a_{4}=0.545130,\qquad b_{4}=-0.03477\quad\text
\]

mit \(U_{\text{konst}}=0.48\). Hier zeigt sich ein numerisch auffälliges Ergebnis.Der Exponent \(b_{3}\) ist nahe Null deutet auf einen konstanten Fit in diesem Bereich hin. 
Die berechneten Grenzfrequenzen ergeben in diesem Fall \(\nu_{\text{links}}\approx 123.359739\mathrm{Hz}\) und \(\nu_{\text{rechts}}\approx 38.839297,\mathrm{Hz}\),
wodurch die formale „Bandbreite“ negativ wird (≈ \(-84.5204426\mathrm{Hz}\)). Ein negatives Ergebnis ist physikalisch nicht sinnvoll und weist darauf hin,
dass hier der Fit der linken und rechten Flanke nicht robust gewählt wurde oder dass die Zuordnung von „links“ und „rechts“ vertauscht bzw. der Plateaubereich nicht korrekt identifiziert wurde. 

\paragraph{Plot}
\begin{figure}[htbp]
  \centering
  \includegraphics[width=0.8\textwidth]{Plots/invamp2.png}
  \caption{Plot zur zweiten Messreihe. Die verwendeten Wiederstände betragen $R_1 = \SI{1}{\kilo\ohm}$, $R_2 = \SI{10}{\kilo\ohm}$}
  \label{fig:inv2}
\end{figure}

Für die dritte Messreihe ergaben sich 

\[
a_{5}=1.828665,\qquad b_{5}=-0.014048\quad\text{(linke Flanke)}
\]

und

\[
a_{6}=177.20940,\qquad b_{6}=-0.565788\quad\text{(rechte Flanke)},
\]

mit \(U_{\text{konst}}=1.709\). Daraus folgen die Grenzfrequenzen \(\nu_{\text{links}}\approx 123.6533,\mathrm{Hz}\) und \(\nu_{\text{rechts}}\approx3653.55023,\mathrm{Hz}\) 
sowie eine Bandbreite von etwa 3529.896902\mathrm{Hz}\. Das Ergebnis ist konsistent mit einer linken Flanke, die nur langsam variiert, und einer steileren rechten Flanke, 
die die obere Frequenzbegrenzung des Verstärkers bestimmt. 

\paragraph{Plot}
\begin{figure}[htbp]
  \centering
  \includegraphics[width=0.8\textwidth]{Plots/invamp3.png}
  \caption{Plot zur dritten Messreihe. Die verwendeten Wiederstände betragen $R_1 = \SI{1}{\kilo\ohm}$, $R_2 = \SI{150}{\kilo\ohm}$}
  \label{fig:inv3}
\end{figure}

\subsection{Integrator}
Für den Integrator wurde die Ausgangsamplitude \(U_{A}\) in Abhängigkeit von der Anregungsfrequenz \(\nu\) aufgenommen. 
Als Bauelemente kamen ein Kondensator \(C=\SI{100}{\nano\farad}\) und ein Wiederstand \(R=\SI{10}{\kilo\ohm}\) zum Einsatz. 
Die gemessenen Verstärkungswerte wurden mit einer Potenzfunktion der Form


\[
V(\nu)=a_{\text{int}}\;\nu^{\,b_{\text{int}}}
\]

angepasst. Die Regression liefert die Parameter

\[
a_{\text{int}} = 132.47 \pm 592
b_{\text{int}} = -0.6519\pm 0.033
\]
ein

Aus dem Fit‑Parameter \(a_{\text{int}}\) und den Bauelementwerten wurde gemäß der im Versuch verwendeten Herleitung die konstante Größe \(C_{\text{konst}}\) berechnet.


\[
C_{\text{konst,int}} = 0.132 F
\]



Die Messdaten und die Fitkurve sind in Abbildung\ref{fig:4} dargestellt. 
Zusätzlich wurde untersucht, welches Verhalten der Integrator bei angelegten Rechteck‑ und Dreieckssignalen zeigt. Wie erwartet führt ein Rechtecksignal 
am Ausgang näherungsweise zu einer dreieckförmigen Spannungsform  während ein Dreieckssignal am Ausgang näherungsweise eine parabolische Form ergibt.

\paragraph{Plot}
\begin{figure}[htbp]
  \centering
  \includegraphics[width=0.8\textwidth]{Plots/integrator.png}
  \caption{Plot zum Integrator, zu sehen sind die Messwerte und die Fitfunktion.}
  \label{fig:4}
\end{figure}

\subsection{Differentiator}
Die Messreihe für den Differentiator wurde analog durchgeführt. Verwendet wurden ein Kondensator \(C=\SI{2}{\nano\farad}\) und ein
 Widerstand \(R=\SI{100}{\kilo\ohm}\). Die Amplitudenfrequenzabhängigkeit wurde ebenfalls mit der Potenzform


\[
V(\nu)=a_{\text{diff}}\;\nu^{\,b_{\text{diff}}}
\]

angepasst. Die Fitparameter ergeben sich zu

\[
a_{\text{diff}} = 0.02049 \pm 0.001
b_{\text{diff}} = 0.90430 \pm 0.010
\]


Für den Differentiator wurde analog die konstante Größe \(C_{\text{konst}}\) berechnet.

\[
C_{\text{konst,diff}} = 4.507e-7 F
\]

\paragraph{Plot}
\begin{figure}[htbp]
  \centering
  \includegraphics[width=0.8\textwidth]{Plots/differentiator.png}
  \caption{Plot der Messwerte sowie einer Ausgleichsgeraden zum Differentiator,}
  \label{fig:diff}
\end{figure}








