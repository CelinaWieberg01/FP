\section{Auswertung}
\label{sec:Auswertung}
Für die Auswertung wird die \texttt{Python}-Bibliothek \texttt{numpy} \cite{numpy} benutzt. Die Fits entstehen mit \texttt{curve\_fit} aus \texttt{scipy.optimize} \cite{scipy}.
Die Fehlerrechnung wird mit \texttt{uncertainties} \cite{uncertainties} durchgeführt. Plots entstehen mit \texttt{matplotlib.pyplot} \cite{matplotlib}.

In abgebildeten Aufnahmen des Oszilloskops ist in gelb das Eingangssignal und in grün das Ausgangssignal dargestellt.



\subsection{Inverting-Amplifier}
Für den Inverting-Amplifier werden verschiedene Verhältnisse von Widerständen verwendet.
Es wird das Verhältnis $A$ zwischen gemessener Ausgangsspannung $U_\text{out}$ und eingestellter Eingangsspannung $U_\text{in}$ gegen die Frequenz in einem doppellogarithmischen Plot dargestellt
und die abnehmende Flanke mit einer Potenzfunktion
\begin{equation*}
    A(f) = a \left(\frac{f}{\SI{1}{\hertz}}\right)^b
\end{equation*}
gefittet. Das annähernd konstante Plateau bei $A(f) = A_0$ wird durch die Messwerte gemittelt und mit dem Erwartungswert von \autoref{eq:inverter} verglichen.
Die Cutoff-Frequenz $f_\text{c}$ ergibt sich durch
\begin{equation*}
    \frac{A_0}{\sqrt{2}} = a \left(\frac{f_\text{c}}{\SI{1}{\hertz}}\right) ^b.
\end{equation*}

Die Plots für die drei verschiedenen Widerstandsverhältnise befinden sich in \autoref{fig:inv1}, \autoref{fig:inv2} und \autoref{fig:inv3}.
Die Fitparameter, Cutoff-Frequenz und Verstärkungen finden sich in \autoref{tab:inv}

\begin{figure}[H]
    \centering
    \includegraphics[width=0.8\textwidth]{plots/inv1.pdf}
    \caption{Inverting-Amplifier mit $R_1 = \SI{1}{\kilo\ohm}$ und $R_2 = \SI{100}{\kilo\ohm}$ und eingezeichneten Fits.}
    \label{fig:inv1}
\end{figure}
\autoref{fig:inv1} zeigt eine konstante Verstärkung bis in den $\si{\kilo\hertz}$-Bereich und folgt einem Potenzgesetz ab circa $\SI{10}{\kilo\hertz}$.

\begin{figure}[H]
    \centering
    \includegraphics[width=0.8\textwidth]{plots/inv2.pdf}
    \caption{Inverting-Amplifier mit $R_1 = \SI{1}{\kilo\ohm}$ und $R_2 = \SI{10}{\kilo\ohm}$ und eingezeichneten Fits.}
    \label{fig:inv2}
\end{figure}
\autoref{fig:inv2} zeigt eine konstante Verstärkung bis in den $\si{\kilo\hertz}$-Bereich sowie auf einer anderen Höhe bis zum $\SI{10}{\kilo\hertz}$-Bereich. Das Potenzgesetz greift hier weniger deutlich, wird für Vergleichbarkeit dennoch durchgeführt.

\begin{figure}[H]
    \centering
    \includegraphics[width=0.8\textwidth]{plots/inv3.pdf}
    \caption{Inverting-Amplifier mit $R_1 = \SI{1}{\kilo\ohm}$ und $R_2 = \SI{150}{\kilo\ohm}$ und eingezeichneten Fits.}
    \label{fig:inv3}
\end{figure}
\autoref{fig:inv3} zeigt eine konstante Verstärkung bis über den $\si{\kilo\hertz}$-Bereich hinaus und folgt einem Potenzgesetz ab circa $\SI{10}{\kilo\hertz}$.

\begin{table}[H]
    \centering
    \caption{Widerstand $R_2$, weil $R_1$ konstant ist, theoretische Verstärkung $A_\text{theo}$, gemittelte Verstärkung $A_0$, Abweichung der Verstärkung $\Delta A$, Fitparameter $a$ und $b$ und Cutoff-Frequenz $f_\text{c}$ verschiedener Widerstandsverhältnisse.}
    \label{tab:inv}
    \begin{tabular}{c c c c c c c}
        \toprule
        {$R_2$ in $\si{\kilo\ohm}$} & {$A_\text{theo}$} & {$A_0$} & {$\Delta A$} & {$a$} & {$b$} & {$f_\text{c}$ in $\si{\hertz}$} \\
        \midrule
        \num{100} & \num{100} & \num{88.0(7)}  & \num{12} \% & \num{5.06(1.19)e4} & \num{-0.719(25)} & \num{1.1(5)e4}  \\
        \num{10}  & \num{10}  & \num{11.2(1)}  & \num{12} \% & \num{33.1(11.5)} & \num{-0.132(35)} &   \num{0.5(1.9)e5}  \\
        \num{150} & \num{150} & \num{106.0(4)} & \num{29} \% & \num{4.85(1.06)e4} & \num{-0.72(2)} &   \num{8.0(3.3)e3}  \\
        \bottomrule
    \end{tabular}
\end{table}

Wegen der Konstanz der ersten Messwerte wird auf einen Fit mit einem Potenzgesetz verzichet. Die untere Cutoff-Frequenz wird somit auf $0$ gesetzt,
sodass die Bandbreite gleich $f_\text{c}$ ist.



\newpage
\subsection{Integrator}
Für den Integrator wird ebenfalls die Verstärkung $A$ wie für den Inverting-Amplifier ermittelt.
Gemäß \autoref{eq:integrator} wird $f$ in $\omega$ umgerechnet und $A$ gegen das inverse $\omega$ geplottet. Mit einem linearen Fit der Form
\begin{equation}
    A(f) = m \cdot \frac{1}{\omega} + b,
    \label{eq:linfit}
\end{equation}
wobei $m = \frac{1}{RC}$ in $\si{\per\second}$ der inversen Zeitkonstanten $RC$ entspricht, lässt sich die Zeitkonstante ermitteln. Dies geschieht in
\autoref{fig:int}

\begin{figure}[H]
    \centering
    \includegraphics[width=0.8\textwidth]{plots/int.pdf}
    \caption{Verstärkungsfaktoren gegen die inverse Frequenz mit Fit zur Bestimmung der Zeitkonstante bei $R = \SI{100}{\kilo\ohm}$ und $C = \SI{100}{\nano\farad}$.}
    \label{fig:int}
\end{figure}

Die ermittelten Fitparameter lauten
\begin{align*}
    m &= \qty{1.03(0.03)e3}{\per\second} \\
    b &= \num{2.34(26)}
\end{align*}

und ergeben eine Zeitkonstante von 
\begin{equation*}
    \tau = \qty{0.00097(3)}{\second}.
\end{equation*}

Mit den eingestellten $R = \SI{100}{\kilo\ohm}$ und $C = \SI{22}{\nano\farad}$ ergibt sich eine theoretische Zeitkonstante von
\begin{equation*}
    \tau_\text{Integrator} = RC = \qty{0.001}{\second}.
\end{equation*}

Die Abweichung zwischen Theoriewert und Messwert lautet
\begin{equation*}
    \Delta \tau = \qty{2.6}{\percent}.
\end{equation*}

Zur Demonstration, dass der Integrator als dieser fungiert, sind in \autoref{fig:int_sin}, \autoref{fig:int_recht} und \autoref{fig:int_drei}
jeweils eine Sinus-, Rechteck- und Dreieckspannung als Eingang und ihr integrierter Output dargestellt.

\begin{figure}[H]
    \centering
    \includegraphics[width=0.5\textwidth]{data/bilder/integrator_sinus.png}
    \caption{Sinusspannung und integrierte Sinusspannung.}
    \label{fig:int_sin}
\end{figure}

\begin{figure}[H]
    \centering
    \includegraphics[width=0.5\textwidth]{data/bilder/integrator_rechteck.png}
    \caption{Rechteckspannung und integrierte Rechteckspannung.}
    \label{fig:int_recht}
\end{figure}

\begin{figure}[H]
    \centering
    \includegraphics[width=0.5\textwidth]{data/bilder/integrator_dreieck.png}
    \caption{Dreieckspannung und integrierte Dreieckspannung.}
    \label{fig:int_drei}
\end{figure}



\newpage
\subsection{Differentiator}
Für den Differentiator wird ebenfalls die Verstärkung $A$ wie für den Inverting-Amplifier ermittelt.
Gemäß \autoref{eq:differentiator} wird $f$ in $\omega$ umgerechnet und $A$ gegen $\omega$ geplottet. Mit einem linearen Fit wie in \autoref{eq:linfit}
wobei jetzt $m = RC$ in $\si{\second}$ der Zeitkonstanten $RC$ entspricht, lässt sich die Zeitkonstante ermitteln. Dies geschieht in
\autoref{fig:dif}

\begin{figure}[H]
    \centering
    \includegraphics[width=0.8\textwidth]{plots/diff.pdf}
    \caption{Verstärkungsfaktoren gegen die Frequenz mit Fit zur Bestimmung der Zeitkonstante bei $R = \SI{100}{\kilo\ohm}$ und $C = \SI{22}{\nano\farad}$.}
    \label{fig:dif}
\end{figure}

Die ermittelten Fitparameter lauten
\begin{align*}
    m &= \qty{1.85(2)e-3}{\second} \\
    b &= \num{0.139(15)}
\end{align*}

und ergeben eine Zeitkonstante von 
\begin{equation*}
    \tau = \qty{0.00185(1)}{\second}.
\end{equation*}

Mit den eingestellten $R = \SI{100}{\kilo\ohm}$ und $C = \SI{22}{\nano\farad}$ ergibt sich eine theoretische Zeitkonstante von
\begin{equation*}
    \tau_\text{Integrator} = RC = \qty{0.0022}{\second}.
\end{equation*}

Die Abweichung zwischen Theoriewert und Messwert lautet
\begin{equation*}
    \Delta \tau = \qty{16}{\percent}
\end{equation*}

\newpage
Zur Demonstration, dass der Differentiator als dieser fungiert, sind in \autoref{fig:dif_sin}, \autoref{fig:dif_recht} und \autoref{fig:dif_drei}
jeweils eine Sinus-, Rechteck- und Dreieckspannung als Eingang und ihr differenzierter Output dargestellt.

\begin{figure}[H]
    \centering
    \includegraphics[width=0.5\textwidth]{data/bilder/differentiator_sinus.png}
    \caption{Sinusspannung und differenzierte Sinusspannung.}
    \label{fig:dif_sin}
\end{figure}

\begin{figure}[H]
    \centering
    \includegraphics[width=0.5\textwidth]{data/bilder/differentiator_rechteck.png}
    \caption{Rechteckspannung und differenzierte Rechteckspannung.}
    \label{fig:dif_recht}
\end{figure}

\begin{figure}[H]
    \centering
    \includegraphics[width=0.5\textwidth]{data/bilder/differentiator_dreieck.png}
    \caption{Dreieckspannung und differenzierte Dreieckspannung.}
    \label{fig:dif_drei}
\end{figure}



\subsection{Schmitt-Trigger}
Die Ausgangssättigungsspannung mit $R_1 = \SI{10}{\kilo\ohm}$ und $R_2 = \SI{100}{\kilo\ohm}$ ergibt sich zu 
\begin{equation*}
    U_\text{S} = \qty{28.1(1)}{\volt}
\end{equation*}
und die Schwellwert-Spannung nach \autoref{eq:schwellwert} lautet in der Theorie
\begin{equation*}
    U_\pm = \qty{2.81(1)}{\volt}.
\end{equation*}

Der Schwellwert wird im Experiment ermittelt und beträgt
\begin{equation*}
    U_\pm = \qty{2.980(1)}{\volt}.
\end{equation*}
Die Abweichung zwischen Theoriewert und Messwert lautet somit
\begin{equation*}
    \Delta U_\pm = \qty{6}{\percent}.
\end{equation*}

Die Aufnahme des hier verwendeten Schmitt-Triggers befindet sich in \autoref{fig:schm}.
\begin{figure}[H]
    \centering
    \includegraphics[width=0.8\textwidth]{data/bilder/scope_6.png}
    \caption{Aufnahme vom Oszilloskop bei ausgelöstem Schmitt-Trigger.}
    \label{fig:schm}
\end{figure}



\newpage
\subsection{Generator}
Vom ersten Signalgenerator kann die Frequenz gemäß \autoref{eq:frequenz} berechnet werden.
Diese beläuft sich zu 
\begin{equation*}
    f = \qty{34}{\percent}.
\end{equation*}

Die Messung ist dargestellt in \autoref{fig:gen1} und die dort gemessene Frequenz lautet 
\begin{equation*}
    f = \SI{1.65(10)e3}{\hertz}
\end{equation*}

Die Abweichung zwischen Theoriewert und Messwert lautet
\begin{equation*}
    \Delta f = \qty{34}{\percent}.
\end{equation*}

\begin{figure}[H]
    \centering
    \includegraphics[width=0.8\textwidth]{data/bilder/scope_7.png}
    \caption{Generator, der eine Rechteckspannung eines Schmitt-Triggers integriert.}
    \label{fig:gen1}
\end{figure}


\subsection{Generator mit variierender Amplitude}
Für die Schwingdauer $T$ und Abklingzeit $\tau$ des dämpfbaren Oszillators ergeben sich nach \autoref{eq:T} und \autoref{eq:tau} mit $\eta = \pm 1$
\begin{align*}
    T &= \SI{0.00628}{\second} \\
    \tau &= \SI{0.02}{\second}
\end{align*}

Die Aufnahmen vom Oszilloskop befinden sich in \autoref{fig:gen2_scope_max} und \autoref{fig:gen2_scope_min}.

\begin{figure}[H]
    \centering
    \includegraphics[width=0.6\textwidth]{data/bilder/scope_11.png}
    \caption{Oszilloskopaufnahme bei maximaler Dämpfung.}
    \label{fig:gen2_scope_max}
\end{figure}

\begin{figure}[H]
    \centering
    \includegraphics[width=0.6\textwidth]{data/bilder/scope_12.png}
    \caption{Oszilloskopaufnahme bei minimaler Dämpfung.}
    \label{fig:gen2_scope_min}
\end{figure}

Ein Fit nach \autoref{eq:loesung} ist nicht möglich.
Um sinnvolle Werte zu ermitteln, werden die Messdaten verkürzt und transformiert.
Die Messung in \autoref{fig:gen2_max} wird auf alle Datenpunkte mit $U > \num{0.03}$ und dann mit $t < \num{0.3}$ gekürzt, für \autoref{fig:gen2_min} ist $t < \num{0.28}$, 
anschließend wird $U$ um den Mittelwert in diesem Intervall verschoben und der Datensatz in $t$ auf $0$ gesetzt. Mit \texttt{scipy.signal.find\_peaks}
werden die Peaks der Oszillationen rausgesucht. Aufgrund der zitternden Oszillationen wird den Peaks in $U$ und $t$ ein Fehler von $\SI{0.0015}{\volt}$ bzw. $\si{\second}$ zugeschrieben.

Die transformierten Messpunkte befinden sich in \autoref{fig:gen2_max} und \autoref{fig:gen2_min}.

\begin{figure}[H]
    \centering
    \includegraphics[width=0.8\textwidth]{plots/gen2_max_sec.pdf}
    \caption{Transformierte Messdaten und eingezeichnete gefundene Peaks bei maximaler Dämpfung.}
    \label{fig:gen2_max}
\end{figure}

\begin{figure}[H]
    \centering
    \includegraphics[width=0.8\textwidth]{plots/gen2_min_sec.pdf}
    \caption{Transformierte Messdaten und eingezeichnete gefundene Peaks bei minimaler Dämpfung.}
    \label{fig:gen2_min}
\end{figure}

Um an $T$ zu kommen, werden die Abstände vom positiven zum positiven bzw. negativen zum negativen Peak gemittelt.
Um an $\tau$ zu kommen, werden die negativen Peaks mit $-1$ multipliziert und ihr Abfall mit 
\begin{equation*}
    U(t) = U_0 e^{-\frac{t}{\tau}}
\end{equation*}
gefittet.

Die errechneten und gefundenen Periodendauern, Abklingzeiten, $U_0$ und Abweichungen sind in \autoref{tab:generators} dargestellt.

\begin{table}[H]
    \centering
    \caption{}
    \label{tab:}
    \begin{tabular}{c c c c c c c c}
        \toprule
        {Dämpfung} & {$T_\text{theo}$ in $\si{\second}$} & {$T$ in $\si{\second}$} & {$\Delta T$} & {$\tau_\text{theo}$ in $\si{\second}$} & {$\tau$ in $\si{\second}$} & {$\Delta \tau$} & {$U_0$ in $\si{\volt}$} \\
        \midrule
        $+1$ & \num{0.00628} & \num{0.00536(16)} & \num{15} \% & \num{0.02} & \num{0.081(25)} & \num{310} \% & 0.011(1) \\
        $-1$ & \num{0.00628} & \num{0.00542(25)} & \num{14} \% & \num{0.02} & \num{0.027(6)}  & \num{38}  \% & 0.010(1) \\
        \bottomrule
    \end{tabular}
\end{table}