\section{Durchführung}
\label{sec:Durchführung}
Im Folgenden wird der grundsätzliche Aufbau des Versuches sowie die Durchführung der einzelnen Aufgaben beschrieben.

\subsection{Versuchsaufbau}
Der Versuch wird auf einem Breadboard durchgeführt, dargestellt in \autoref{fig:breadboard}, die Verbindungen der Anschlüsse sind in \autoref{fig:connections} visualisiert. Des Weiteren stehen eine Spannungsquelle, ein Signalgenerator und ein Oszilloskop zur Verfügung.
Kleine Kupferkabel können verschiedene elektrische Bauteile miteinander verbinden. Diese sind alle in \autoref{fig:devices} zu sehen.

\begin{figure}[H]
    \centering
    \includegraphics[width=0.75\textwidth]{theorie_bilder/bredboard.jpg}
    \caption{Das verwendete Breadboard. Das blaue Kabel links bzw. das rote Kabel rechts sind die negativen bzw. positiven Enden der Betriebsspannung. Das schwarze Kabel oben ist die Erde. Die Kabel unten links und rechts greifen das Eingangs- und Ausgangssignal ab.
    In der Mitte in Schwarz befindet sich ein OP-Amp \cite{anleitung}.}
    \label{fig:breadboard}
\end{figure}

\begin{figure}[H]
    \centering
    \includegraphics[width=0.7\textwidth]{theorie_bilder/connections.png}
    \caption{Visualisierung der Verbindungen in einem Breadboard. Die $+$- und $-$-Spalten verbinden durchgehend, während die Anschlüsse in einer Zeile nur von A bis E bzw. F bis J verbinden. Entnommen aus \cite{anleitung}.}
    \label{fig:connections}
\end{figure}


\begin{figure}[H]
    \centering
    \includegraphics[width=0.7\textwidth]{theorie_bilder/devices.png}
    \caption{Weitere Geräte, die im Versuch verwendet wurden. Von links nach rechts, von oben nach unten: Signalgenerator, Oszilloskop, Spannungsquelle, Kasten mit elektrischen Bauteilen, Breadboard, Multimeter \cite{anleitung}.}
    \label{fig:devices}
\end{figure}


\subsection{Versuchsdurchführung}

Die Betriebsspannungen werden für den gesamten Versuch auf $\pm\SI{15}{\volt}$ geregelt. Die Frequenzen und Amplituden der Eingangssignale können wenn nötig mit dem Signalgenerator geregelt werden,
der Sinus-, Rechteck- und Dreieckspannungen bereitstellen kann. Die Frequenzen und Amplituden der Ausgangsspannungen können am Oszilloskop ausgelesen werden.


\subsubsection{Inverting-Amplifier}
Der Inverting-Amplifier wird gemäß \autoref{fig:inv_ampl} aufgebaut. Es wird ein Sinus-Signal eingespeist, dessen Frequenz variiert wird. Die Amplitude des Ausgangssignals wird gemessen.
Es werden drei Messungen für den Inverting-Amplifier durchgeführt, jeweils mit verschiedenen Widerständen. Die erste Messung wird durchgeführt mit $R_1 = \SI{1}{\kilo\ohm}$ und $R_2 = \SI{100}{\kilo\ohm}$, die zweite Messung mit  $R_1 = \SI{1}{\kilo\ohm}$ und $R_2 = \SI{10}{\kilo\ohm}$
und die dritte Messung mit  $R_1 = \SI{1}{\kilo\ohm}$ und $R_2 = \SI{150}{\kilo\ohm}$.

\subsubsection{Integrator}
Der Integrator wird gemäß \autoref{fig:integrator} aufgebaut, der Widerstand beträgt $R = \SI{10}{\kilo\ohm}$ und die Kapazität $C = \SI{100}{\nano\farad}$.
Über den Amplitudenregler am Frequenzgenerator wird die Eingangsamplitude einer Sinusspannung für Frequenzen über verschiedene Dekaden geregelt und die Ausgangsamplitude gemessen.
Anschließend wird für eine Dreiecksspannung und eine Rechtecksspannung ein Bild der integrierten Signale aufgenommen.

\subsubsection{Differentiator}
Die Messung für den Differentiator erfolgt komplett analog wie zum Integrator, nur mit dem Schaltbild aus \autoref{fig:differentiator}, einem Widerstand $R = \SI{100}{\kilo\ohm}$ und einer Kapazität $C = \SI{22}{\nano\farad}$.
Auch hier werden entsprechende Bilder für verschiedene Signalprofile aufgenommen.

\subsubsection{Schmitt-Trigger}
Der Schmitt-Trigger wird gemäß \autoref{fig:schmitt} aufgebaut. Um den Schwellwert zu finden, wird die Amplitude eines Sinussignals langsam und vorsichtig erhöht,
bis diese groß genug ist, um den Zustand des Triggers zu ändern und ein Rechtecksignal ausgegeben wird.

\subsubsection{Generator}
Der Generator wird wie in \autoref{fig:generator1} aufgebaut. Die Größen der verwendeten Bauteile lauten dabei
$R_1 = \SI{10}{\kilo\ohm}$, $R_2 = \SI{100}{\kilo\ohm}$, $R_3 = \SI{1}{\kilo\ohm}$ und $C = \SI{1}{\micro\farad}$.
Wie in der Theorie beschrieben liegt hier ein Schmitt-Trigger gefolgt von einem Integrator vor,
sodass das vom Trigger erzeugte Rechtecksignal integriert wird und ein Dreiecksignal entsteht. Ein Bild wird aufgenommen und die Amplitude wie die Frequenz können abgelesen werden.

\subsubsection{Generator mit variierender Amplitude}
Der Generator wird wie in \autoref{fig:generator2} dargestellt aufgebaut. Die Widerstände stehen im Schaltbild,
als Kapazitäten wird $C = \SI{100}{\nano\farad}$ benutzt. Das Eingangssignal ist eine Rechteckspannung.
Für die beiden extremalen Einstellungen des Potentiometers werden die Messwerte des Oszilloskops aufgezeichnet.