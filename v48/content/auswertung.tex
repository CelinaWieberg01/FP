\section{Auswertung}
\label{sec:Auswertung}

Für die Auswertung wird die \texttt{Python}-Bibliothek \texttt{numpy} \cite{numpy} benutzt. Die Fits entstehen mit \texttt{curve\_fit} aus \texttt{scipy.optimize} \cite{scipy}.
Die Fehlerrechnung wird mit \texttt{uncertainties} \cite{uncertainties} durchgeführt. Plots entstehen mit \texttt{matplotlib.pyplot} \cite{matplotlib}.

\subsection{Fehler}
Der Mittelwert $\bar{x}$ von $N$ gemessenen Werten $a$ bestimmt sich über
\begin{equation}
    \bar{x} = \frac{1}{N} \sum^N_{i=1} a_i,
    \label{eq:mittelwerte}
\end{equation}
der Fehler des Mittelwertes über
\begin{equation}
    \Delta x = \sqrt{\frac{1}{N \cdot (N-1)} \sum^N_{i=1}(a_i - \bar{x})}.
    \label{eq:mittelwerte_fehler}
\end{equation}
Die Gaußsche Fehlerfortpflanzung für eine berechnete Größe $f$ lautet
\begin{equation}
    \Delta f = \sqrt{ \sum^N_{i=1} \left( \frac{\delta f}{\delta x_i}\right)^2 \cdot (\Delta x_i)^2}.
\end{equation}
Prozentuale Abweichungen werden mit
\begin{equation}
    \Delta x = \left|\frac{x - a}{a}\right|
    \label{eq:abweichung}
\end{equation}
berechnet, wobei $a$ ein Vergleichswert und $x$ der erhaltene Wert ist.