\appendix{Emissionsspektren}
\label{sec:Originaldaten}

Die Vergleichsdaten sind dem Laboratoire National Henri Becquerel entnommen.
\begin{table}[H]
    \centering
    \caption{Emissionen von Europium-152. \cite{Europium}}
    \label{tab:LNHB_Eu152}
    \begin{tabular}{c c}
        \toprule
        {Energie ($\si{\kilo\electronvolt}$)} & {Emissionswahrscheinlichkeit ($\%$)} \\
        \midrule
        \num{121.7817(3)} & \num{28.41(13)} \\
        \num{244.6984(8)} & \num{7.44(4)} \\
        \num{344.2785(12)} & \num{26.59(12)} \\
        \num{411.1165(12)} & \num{2.238(10)} \\
        \num{443.965(3)} & \num{2.80(2)} \\
        \num{778.9045(24)} & \num{12.97(6)} \\
        \bottomrule
    \end{tabular}
\end{table}

\begin{table}[H]
    \centering
    \caption{Emissionen für Cäsium-137. \cite{Caesium}}
    \label{tab:LNHB_Cs137}
    \begin{tabular}{c c}
        \toprule
        {Energie ($\si{\kilo\electronvolt}$)} & {Emissionswahrscheinlichkeit ($\%$)} \\
        \midrule
        \num{661.655(3)} & \num{85.01(20)} \\
        \bottomrule
    \end{tabular}
\end{table}

\begin{table}[H]
    \centering
    \caption{Emissionen für Barium-133. \cite{Barium}}
    \label{tab:LNHB_Ba133}
    \begin{tabular}{c c}
        \toprule
        {Energie ($\si{\kilo\electronvolt}$)} & {Emissionswahrscheinlichkeit ($\%$)} \\
        \midrule
        \num{80.9979(11)} & \num{33.31(30)} \\
        \num{276.3989(12)} & \num{7.13(6)} \\
        \num{302.8508(5)} & \num{18.31(11)} \\
        \num{356.0129(7)} & \num{62.05(19)} \\
        \num{383.8485(12)} & \num{8.94(6)} \\
        \bottomrule
    \end{tabular}
\end{table}

