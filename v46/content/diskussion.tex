\section{Diskussion}
\label{sec:Diskussion}

\subsection{Messung der Feldstärke}
Die Messreihe zur Bestimmung der maximalen Feldstärke lief Problemlos ab. Die Sonde war fest 
justiert und ließ sich einfach und genau in den Magneten hinein und hinaus bewegen. Die Werte waren 
an einem digiutalen Messgerät präzise abzulesen und je nach Stärke das Feldes ließ sich die range des 
Gerätes anpassen. Wie man in \ref{fig: Abbildung 1} sehen kann, spiegeln die Messwerte alle erfahrungen wieder.
Das Maximum ist klar abzulesen.

\subsection{Bestimmung der Drehwinkel}
Bei der Bestimmung der Drehwinkel gibt es einige Werte die Stark von den anderen abweichen. Ein möglicher 
Grund dafür ist, dass es nicht immer möglich war das Minimun auf dem Oszilloskop genau abzulesen. Das Signal 
hat Teilweise geflackert oder war nicht scharf in dem Bereich der Minimalen Amplitude, so dass man gezwungen war die Werte 
zu nehmen die stabil angezeigt wurden. Die Werte die deutlich von der Kurve abweichen, beeinflussen die lineare Regression,
so ist bei Probe zwei quasi keine Steigung zu sehen.
