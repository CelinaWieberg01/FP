\section{Diskussion}
\label{sec:Diskussion}

\subsection{Messung der Feldstärke}
Die Messreihe zur Bestimmung der maximalen Feldstärke lief Problemlos ab. Die Sonde war fest 
justiert und ließ sich einfach und genau in den Magneten hinein und hinaus bewegen. Die Werte waren 
an einem digiutalen Messgerät präzise abzulesen und je nach Stärke das Feldes ließ sich die range des 
Gerätes anpassen. Wie man in \ref{fig: Abbildung 1} sehen kann, spiegeln die Messwerte alle erfahrungen wieder.
Das Maximum ist klar abzulesen.

\subsection{Bestimmung der Drehwinkel}
Bei der Bestimmung der Drehwinkel gibt es einige Werte die Stark von den anderen abweichen. Ein möglicher 
Grund dafür ist, dass es nicht immer möglich war das Minimun auf dem Oszilloskop genau abzulesen. Das Signal 
hat Teilweise geflackert oder war nicht scharf in dem Bereich der Minimalen Amplitude, so dass man gezwungen war die Werte 
zu nehmen die stabil angezeigt wurden. Die Werte die deutlich von der Kurve abweichen, beeinflussen die lineare Regression,
so ist bei Probe zwei quasi keine Steigung zu sehen.

\subsection{Bestimmung der Effektiven Masse}
Der theoretische Wert der effektiven masse in Galliumarsenid beträgt 0.067 $m_e$. Vergleicht man diesen Wert mit den
experimentell bestimmten Massen, ist eine signifikante Abweichung zu sehen. Die Prozentuale Abweichungg berechnet sich nach:
\begin{equation} 
    \text{diff} = \frac{(m_{\text{literatur}} - m_{\text{exp}})}{m_{\text{exp}}} \cdot 100 
\end{equation}
Die erste Masse hat einen Wert von 0.3073$m_e$, was einer Abweichung von 358\% entspricht.
Die zweite Masse hat einen Wert von 0.7941$m_e$, was einer Abweichung von 1063\% entspricht.
Grund für diese Ungenauigkeiten können Schwierigkeiten beim genauen Ablesen gewesen sein, sowie 
eine nicht optimale Justierung. Trotz der deutlichen Abweichung von dem Literaturwert, ist
in \ref{fig:regression } eine lineare, positive Steigung zu erkennen, was den Erwartungen entspricht.




