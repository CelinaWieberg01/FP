\section{Diskussion}
\label{sec:Diskussion}

\subsection{Messung der Feldstärke}
Die Messreihe zur Bestimmung der maximalen Feldstärke verlief Problemlos. Die Sonde war fest 
justiert und ließ sich einfach und genau in den Magneten hinein und hinaus bewegen. Die Werte waren 
an einem Teslameter präzise abzulesen und je nach Stärke das Feldes ließ sich die Range des 
Gerätes anpassen. Eine Verbesserung der Messung wäre machbar, indem die genaue Position der Proben präziser vermessen werden.

\subsection{Bestimmung der Drehwinkel}
Bei der Bestimmung der Drehwinkel gibt es einige Werte die Stark von den anderen abweichen. Ein möglicher 
Grund dafür ist, dass es nicht immer möglich war das Minimum auf dem Oszilloskop genau abzulesen. Das Signal 
hat Teilweise geflackert oder war nicht scharf in dem Bereich der minimalen Amplitude, sodass zwangszweise die Werte 
zu nehmen sind, die stabil angezeigt werden. Die Werte die deutlich von der Kurve abweichen, beeinflussen die lineare Regression,
so sieht man in \autoref{fig:regression1} und \autoref{fig:regression2}, dass die Fits unter Auslassen des vierten Messwertes möglicherweise näher an den Messwerten
liegen könnten.

\subsection{Bestimmung der Effektiven Masse}
<<<<<<< HEAD
Der theoretische Wert der effektiven masse in Galliumarsenid lautet $0, 067 \cdot m_\text{e}$. Die Abweichungen gemäß \autoref{eq:abweichung} betragen
||||||| 05b9994
Der theoretische Wert der effektiven masse in Galliumarsenid lautet 0.067$cdot m_\text{e}$. Die Abweichungen gemäß \autoref{eq:abweichungen} betragen
=======
Der theoretische Wert der effektiven masse in Galliumarsenid lautet $\num{0.067} \cdot m_\text{e}$. Die Abweichungen gemäß \autoref{eq:abweichung} betragen
>>>>>>> 488e4e5413ead7ea746f355c1424b3fbb286a0b0

\begin{align*}
<<<<<<< HEAD
    \Delta(\text{Probe 1, konst. n}) &= 67.4 \, \% \\
    \Delta(\text{Probe 2, konst. n}) &= 64.7 \, \% \\
    \Delta(\text{Probe 1, var. n})   &= 82.2 \, \% \\
    \Delta(\text{Probe 2, var. n})   &= 81.2 \, \% 
||||||| 05b9994
    \Delta(\text{Probe 1, konst. n}) &= 375 \, \% \\
    \Delta(\text{Probe 2, konst. n}) &= 538 \, \% \\
    \Delta(\text{Probe 1, var. n})   &= 162 \, \% \\
    \Delta(\text{Probe 2, var. n})   &= 260 \, \% 
=======
    \Delta(\text{Probe 1, konst. n}) &= 97 \, \% \\
    \Delta(\text{Probe 2, konst. n}) &= 97 \, \% \\
    \Delta(\text{Probe 1, var. n})   &= 98 \, \% \\
    \Delta(\text{Probe 2, var. n})   &= 98 \, \% 
>>>>>>> 488e4e5413ead7ea746f355c1424b3fbb286a0b0
\end{align*}
Die Abweichungen sind unvernachlässigbar groß. Mögliche Gründe dafür sind Probleme in der Bestimmung der Minima des Differenzenverstärkers, wie oben beschrieben.
Des Weiteren ist zu beachten, dass die effektive Masse laut \autoref{eq:effektive_masse} nicht konstant sein muss. Der hier referenzierte Wert
ist die effektive Masse im Energieminimum, sodass nicht abwägig ist, davon auszugehen, dass betrachteten Elektronen diese effektive Masse haben sollten. Da der Versuch allerdings ungekühlt stattfindet und das Magnetfeld
die Proben merklich erhitzt, ist es denkbar, dass auch andere Symmetriepunkte beteiligt sind, die eine viel höhere effektive Masse besitzen. Eine Verbesserung wäre also mit einer Kühlung umsetzbar.\\
Anhang der Abweichungen sieht man, dass die Berücksichtigung der variablen Brechungsindizes genauere Berechnungen möglich sind.

