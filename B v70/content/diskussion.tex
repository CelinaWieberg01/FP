\section{Diskussion}
\label{sec:Diskussion}

\subsection{Saugvermögen}

Für die Drehschieberpumpe werden folgende Saugvermögen bestimmt und in \autoref{fig:abweichungen_dp} dargestellt.
\begin{align*}
    S_1 &= (\num{1.39} \pm \num{0.14}) \si{\liter\per\second} \\
    S_2 &= (\num{0.49} \pm \num{0.05}) \si{\liter\per\second} \\
    S_3 &= (\num{0.41} \pm \num{0.04}) \si{\liter\per\second} \\
    S_{100} &= (\num{1.54} \pm \num{0.16}) \si{\liter\per\second} \\
    S_{50} &= (\num{1.64} \pm \num{0.17}) \si{\liter\per\second} \\
    S_{10} &= (\num{1.60} \pm \num{0.17}) \si{\liter\per\second} \\
    S_{\num{0.5}} &= (\num{1.02} \pm \num{0.15}) \si{\liter\per\second}.
\end{align*}

Der Verlust im Saugvermögen bei der Evakuierung, also bei $S_1$, $S_2$ und $S_3$, ist damit zu erklären, dass das Saugvermögen selber druckabhängig ist.
Das vom Hersteller angegebene Saugvermögen beläuft sich auf $\SI{4.6}{\meter\cubed\per\hour}$ \cite{delta_tu_dortmund}, was  $\SI{1.28}{\liter\per\second}$ entspricht.
Die Abweichung vom Mittelwert der Saugvermögen,

\begin{align*}
    \bar{S}_{\text{DP}} = \qty{1.156(442)}{\liter\per\second}
\end{align*}

beträgt $\Delta S = \num{9.7} \, \%$.

\begin{figure}[H]
    \centering
    \includegraphics[width=\textwidth]{Diskussion_DP_Saugvermögen.pdf}
    \caption{Ermittelte Saugvermögen $S$ der Drehschieberpumpe und der berrechnete Mittelwert $\bar{S}_{DP}$.}
    \label{fig:abweichungen_dp}
\end{figure}

Für die Turbomolekularpumpe werden folgende Saugvermögen bestimmt und in \autoref{fig:abweichungen_tp} dargestellt. 
\begin{align*}
    S_1 &= (\num{7.5} \pm \num{0.8}) \si{\liter\per\second} \\
    S_2 &= (\num{0.78} \pm \num{0.08}) \si{\liter\per\second} \\
    S_3 &= (\num{0.192} \pm \num{0.019}) \si{\liter\per\second} \\
    S_{10^{-4}} &= (\num{15} \pm \num{5}) \si{\liter\per\second} \\
    S_{2 \cdot 10^{-4}} &= (\num{23} \pm \num{8}) \si{\liter\per\second} \\
    S_{5 \cdot 10^{-5}} &= (\num{7.5} \pm \num{2.4}) \si{\liter\per\second}\\
    S_{7 \cdot 10^{-5}} &= (\num{10} \pm \num{3.2}) \si{\liter\per\second}.
\end{align*}

\begin{figure}[H]
    \centering
    \includegraphics[width=\textwidth]{Diskussion_TP_Saugvermögen.pdf}
    \caption{Ermittelte Saugvermögen $S$ der Turbomolekularpumpe und der berrechnete Mittelwert $\bar{S}_{TP}$.}
    \label{fig:abweichungen_tp}
\end{figure}

Die Abweichung vom Mittelwert der Saugvermögen,

\begin{align*}
    \bar{S}_{\text{DP}} = \qty{9.13(5.88)}{\liter\per\second}
\end{align*}

zum Literaturwert des Herstellers von $\SI{77}{\liter\per\second}$ \cite{delta_tu_dortmund} beträgt $\Delta S = \num{88} \, \%$.

Diese Unterschiede sind damit zu erklären, dass der Hersteller diese unter Idealbedingungen ermittelt und sie, zumindest im Falle der Turbomolekularpumpe,
für ein bestimmtes Gas, hier Stickstoff, angegeben werden. Diese Idealbedingungen sind unter anderem ein angepasstes, zu evakuierendes Volumen, Temperaturkontrolle und Stoffkontrolle.
Wasserdampf in der Luft ist in der Lage, im Rezipienten zu kondensieren und zu desorbieren, insbesondere wenn die Temperatur über den Verlauf der Messung steigt.
Unter nicht-idealen Bedingungen kommen insbesondere die Desorption bei kleinen Drücken zum Tragen, weshalb die Messwerte für die Drehschieberpumpe viel näher bei den Herstellerwerten liegen als für
die Turbomolekularpumpe, die bei viel kleineren Drücken arbeitet. 

\subsection{Leitwert}

Wird zwischen den zwei Messgeräten ein weiteres Rohr eingefügt, ändert sich die Geometrie des zu evakuierenden Volumens.
Die Detektoren messen unterschiedliche Drücke, wie in \ref{fig:Leitparam_2e4} und \ref{fig:Leitparam_5e5} zu sehen ist, weil sich der Leitwert zwischen den beiden Messgeräten ändert.
Der Leitwert des Rohres spielt allerdings nur bei den niedrigeren Drücken eine Wirkung, sobald es zur molekularen Strömung kommt,
da die Teilchen anders als bei der laminaren Strömung im höheren Druckbereich nicht störungsfrei das Rohr durchfließen können, allerdings wird gleichzeitig bei kleinen Drücken die Desorption stärker. \\
