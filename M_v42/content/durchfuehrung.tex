\section{Durchführung}
\label{sec:Durchführung}
Zu Beginn der Versuchsdurchführung sorgt man für saubere Arbeitsbedingungen und trägt durchgehend fusselfreie Handschuhe.
Sichtbare Oberflächen, Probenträger und Tip‑Halter werden mit Ethanol gereinigt, lose Partikel werden entfernt und das Gerät 
wird auf offensichtliche Beschädigungen kontrolliert.
Für die Spitze wird aus dem gewählten Draht eine möglichst scharfe, idealerweise einatomige Spitze hergestellt. Mit einer Zange wird der 
Draht fixiert und mit einer weiteren wird der Draht gedrückt und durch kontrolliertes entlangziehen unter konstantem Druck eine optimale Bruchachse gefunden.
Vor dem Einsetzen in den Tip‑Holder überprüft man die elektrische Verbindung und die mechanische Klemmung, setzt die Spitze vorsichtig ein und schiebt den Tip‑Holder in das STM.
Die Glasabdeckung wird geschlossen, sofern nicht an Anschlüssen gearbeitet wird.
Die Proben werden unmittelbar vor dem Einbau vorbereitet. HOPG wird frisch geschält und mit sauberen Werkzeugen handhabbar gemacht, die Goldprobe wird mit sauberen Pinzetten eingesetzt.
Beide Proben werden plan und zentriert im Probenhalter fixiert, sodass beim Einfahren keine benachbarten Bauteile oder Hebel berührt werden können. Beim Einführen der Spitze in die
vorgesehene Nut von rechts wird darauf geachtet, empfindliche Hebel nicht zu berühren. Die Spitze wird später gerade und ohne Kippen herausgezogen, um mechanische Beschädigungen zu vermeiden.
Man stellt sicher, dass keine offensichtlichen Störquellen vorhanden sind. Die ADVANCE‑Funktion oder der automatische Approach wird gestartet und während des Annäherns beobachtet man die Reflexion
der Spitze auf der Probe als visuelle Abstandskontrolle. Wenn ein messbarer Tunnelstrom registriert wird, wird das Feedback aktiviert und die Scannersteuerung eingeschaltet.
Für HOPG werden zunächst Übersichtsaufnahmen mit großer Feldgröße aufgenommen, um die allgemeine Bildqualität und die Ebenheit der Probe zu bewerten. Falls die Scanlinie zwar gerade, aber nicht flach ist,
wurde die Probenlage mechanisch nachjustiert. Bei diffuser oder verschwommener Darstellung wurde die Spitze neu aufbereitet, solange bis eine gerade, klare Linie zu erkennen ist.
Anschließend werden schrittweise Vergrößerungen vorgenommen. Es werden vier bis acht hochaufgelöste Bilder aus verschiedenen Regionen der Probe aufgenommen, und für jede Aufnahme werden Bias, Setpoint, Scanrate, Pixelanzahl und Scanrichtung dokumentiert.
Für die Goldprobe erfolgt die Annäherung und Bildaufnahme analog, wobei besonderer Wert auf die Speicherung der Topographie (Z‑Daten) gelegt wird, da die Auswertung auf Höhensprüngen und Terrassen beruht.
Geeignete Setpoints und Scanraten werden entsprechend den Gerätevorgaben gewählt.
Während der Messung werden die Auswirkungen der PID‑Regelung beobachtet und bei Bedarf angepasst, damit der Regelkreis stabil arbeitet.
Alle Rohdaten werden unmittelbar gesichert, systematisch benannt und in einer klaren Ordnerstruktur abgelegt.
Nach Abschluss der Messungen fährt man den Scanner vollständig zurück, stellt die ADVANCE‑Funktion zurück, entfernt die Spitze vorsichtig und
entnimmt die Proben. Das Instrument wird abgedeckt und der Arbeitsbereich gereinigt.

\begin{figure}[htbp]
    \centering
    \includegraphics[width=0.8\textwidth]{pfad/zum/bild.ext}
    \caption{Der Versuchsaufbau inklusive eingebauter Probe.}
    \label{fig:stmbeispiel}
  \end{figure}
