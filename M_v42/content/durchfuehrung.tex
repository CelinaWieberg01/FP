\section{Durchführung}
\label{sec:Durchführung}
Zu Beginn der Versuchsdurchführung wird für saubere Arbeitsbedingungen gesorgt und durchgehend fusselfreie Handschuhe getragen.
Sichtbare Oberflächen, Probenträger und Tip‑Halter werden mit Ethanol gereinigt, lose Partikel werden entfernt und das Gerät 
wird auf offensichtliche Beschädigungen kontrolliert.
Für die Spitze wird aus einem Platin-Iridium-Draht eine möglichst scharfe, idealerweise einatomige, ungefähr \SI{1}{\centi\meter} lange Spitze hergestellt. Mit einer Zange wird der 
Draht fixiert und mit einer weiteren wird der Draht gedrückt und durch kontrolliertes Entlangziehen unter konstantem Druck eine optimale Bruchachse gefunden.
Vor dem Einsetzen in den Tip‑Holder werden die elektrische Verbindung und die mechanische Klemmung überprüft, die Spitze vorsichtig eingesetzt und der Tip‑Holder in das STM geschoben.
Die Glasabdeckung wird geschlossen, sofern nicht an Anschlüssen gearbeitet wird. Die benutzten Gegenstände und der Aufbau des Mikroskops sind in \autoref{fig:aufbau} dargestellt.

\begin{figure}[H]
    \centering
    \includegraphics[width=0.8\textwidth]{bildertheorie/aufbau.png}
    \caption{Der Versuchsaufbau inklusive eingebauter Probe.}
    \label{fig:aufbau}
  \end{figure}

Beide Proben werden plan und zentriert im Probenhalter fixiert, sodass beim Einfahren keine benachbarten Bauteile oder Hebel berührt werden können. Beim Einführen der Spitze in die
vorgesehene Stelle von rechts wird darauf geachtet, empfindliche Hebel nicht zu berühren. Die Spitze wird später gerade und ohne Kippen herausgezogen, um mechanische Beschädigungen zu vermeiden.
Es wird sichergestellt, dass keine offensichtlichen Störquellen vorhanden sind. Das verwendete Programm des Tunnelmikroskops ist \texttt{Nanosurf Naio}.
Mit der \texttt{ADVANCE}‑Funktion wird die grobe Annäherung gestartet und die Reflexion
der Spitze auf der Probe als visuelle Abstandskontrolle beobachtet. Mit der \texttt{APPROACH}-Funktion wird die Probe selbstständig näher an die Nadel geführt, bis ein Tunnelstrom messbar ist.

Für HOPG werden zunächst Übersichtsaufnahmen mit großer Feldgröße aufgenommen, um die allgemeine Bildqualität und die Ebenheit der Probe zu bewerten. Das Ziel ist eine glatte, nicht unbedingt flache Scanlinie zu erhalten.
Falls diese starke Oszillationen oder Unebenheiten aufweist, wird eine neue Nadel präpariert. 
Anschließend werden schrittweise Vergrößerungen vorgenommen. Sobald der aufgenommene Bereich von $\SI{200}{\nano\meter}$ auf ungefähr $\SI{2.31}{\nano\meter}$ reduziert wird, ist die Struktur von HOPG erkennbar.
Die Einstellungen des PID-Reglers werden variiert und so festgelegt, dass ein klares Bild entsteht. Die hier gewählten Größen betragen $P = 1000$, $I = 2000$ und $D = 0$.
Es werden automatisch durchgehend Messungen durchgeführt und am Ende jene Bilder mit einer klaren Struktur für die Auswertung ausgewählt.


Für die Goldprobe erfolgt die Einstellung etwas grober, das Bild wird auf \SI{160}{\nano\meter} gesetzt und analog Messungen durchgeführt. Hier wird darauf geachtet, Kontraste in der Höhe zu vermessen.

Nach Abschluss der Messungen wird der Scanner vollständig zurückgefahren und
die Proben entnommen. Das Instrument wird abgedeckt und der Arbeitsbereich gereinigt.
