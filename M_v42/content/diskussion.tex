\section{Diskussion}
\label{sec:Diskussion}

\subsection{HOPG}
Die berechneten Gitterkonstanten weisen keine große Streuung auf, weichen aber stark vom Literaturwert von \qty{0.246}{\nano\meter} ab.
Diese Abweichung beträgt \qty{32.3}{\percent}. 

Bessere Ergebnisse könnten mit sauberer präparierten Nadeln erhalten werden. Wie in \autoref{fig:plots0} zu sehen ist,
waren die hier durchgeführten Scans nicht perfekt.
Eine weitere Fehlerquelle ist eine systematische Verzerrung des Scanbildes. In den Fouriertransformationen ist erkennbar, dass die gemessenen Abstände eher flach und breit sind als regelmäßig.
Eine Ursache dafür könnte ein nicht perfektes Alignment der Messspitze auf der Probe sein.

\subsection{Gold}
Die berechnete Schichtdicke beträgt \SI{1.51(9)}{\nano\meter}. Ein Goldatom hat einen Radius von ungefähr \SI{0.144}{\nano\meter} \cite{goldradius}, also einen Durchmesser von \SI{0.288}{\nano\meter} \cite{goldradius}.
Dies deutet darauf hin, dass hier nicht einzelne, ein Atom dicke Schichten vermessen wurden, sondern konsistent ungefähr fünf Atome dicke Schichten.
Da tendenziell Stellen aus der Messung genommen wurden, an denen ein Kontrast optisch erkennbar war, könnte die mehratomige Schichtdicke dadurch erklärt werden.
Eine einatomige Lage hätte einen viel schwächeren Kontrast.

Die Berechnung der Schichtdicke könnte durch bessere Statistik genauer werden. Allerdings decken größere Flächen, die ausgewertet werden, auch mehr mögliche Höhen ab. Das erschwert die Auswertung der Statistik.
Eine Alternative ist, kleinere Bilder aufzunehmen und somit zu versuchen, die Streuung der Höhen zu verringern, sodass schwache Kontraste von einatomigen Schichten nicht unterdrückt werden von starken Kontrasten.