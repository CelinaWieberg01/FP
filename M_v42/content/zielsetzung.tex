\section{Zielsetzung}
\label{sec:Zielsetzung}
Ziel des Versuchs ist das praktische Kennenlernen und die Anwendung des Rastertunnelmikroskops (STM).
Konkret sollen reproduzierbare atomare Aufnahmen der HOPG-Oberfläche erzeugt und die hexagonale Gitterstruktur sichtbar gemacht werden. 
Aus diesen Bildern wird die Gitterkonstante quantitativ bestimmt, einerseits mittels pixelbasierter Auswertung durch Einzeichnen von
Basisvektoren und Kalibrierung, andererseits mittels Fourier-Transformation, außerdem wird ein Vergleich der Methoden einschließlich
Fehlerabschätzung durchgeführt. Für die Goldprobe werden topographische Sprunghöhen aus mehreren Profilen extrahiert und ausgewertet.
Zusätzlich soll die exponentielle Abstandsabhängigkeit des Tunnelstroms qualitativ verifiziert werden und der Einfluss typischer
Störfaktoren identifiziert und minimiert werden. Abschließend werden die Wirkung und sinnvolle Einstellung eines PID-Reglers für 
stabilen Konstantstrom-Betrieb demonstriert.
