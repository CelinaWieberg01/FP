\section{Theorie}
\label{sec:Theorie}
Die Rastertunnelmikroskopie basiert auf den Effekt der Tunnelströme, die im folgenden erklärt werden. Es wird auf den Aufbau und die Funktionsweise von Piezoröhren mit PID-Reglern eingegangen.
Das nötige Hintergrundwissen für die zu vermessenden Proben HOPG und Gold wird gegeben.

\subsection{Tunneleffekt}
Der Tunneleffekt ist ein rein quantenmechanisches Phänomen, bei dem Teilchen eine klassisch unüberwindbare Potentialbarriere mit endlicher Wahrscheinlichkeit durchdringen.
Im Rastertunnelmikroskop führt dies dazu, dass Elektronen zwischen einer metallischen Spitze und der Probenoberfläche durch die Vakuumbarriere tunneln, obwohl ihre Energie
unter der effektiven Barrierenhöhe liegt. Da die Tunnelwahrscheinlichkeit stark vom Abstand zwischen Spitze und Probe abhängt, lässt sich über die Messung des Tunnelstroms 
die Topographie der Oberfläche mit atomarer Auflösung rekonstruieren.
Zur quantitativen Beschreibung wird ein Elektron mit Energie \(E\) und Masse $m$ in einer eindimensionalen Barriereregion mit konstanter effektiver Barrierenhöhe \(\Phi\) über 
der Fermienergie und Breite \(d\) betrachtet. Die zeitunabhängige Schrödingergleichung für die Wellenfunktion $\psi$ in der Barriereregion lautet

\begin{equation*}
-\frac{\hbar^2}{2m}\frac{\mathrm{d}^2\psi}{\mathrm{d}x^2}+(\Phi-E)\psi=0.
\end{equation*}

$\hbar$ ist das Planck'sche Wirkungsquantum. Für \(E<\Phi\) ergibt sich ein reeller Dämpfungsparameter

\begin{equation*}
\kappa=\frac{\sqrt{2m(\Phi-E)}}{\hbar},
\end{equation*}

wodurch die Wellenfunktion in der Barriereregion exponentiell abklingt, \(\psi(x)\propto e^{-\kappa x}\).
Die Transmissionswahrscheinlichkeit durch die Barriere skaliert in führender Näherung wie


\[
T\propto e^{-2\kappa d}.
\]

Bei kleinen Biasspannungen und unter der Annahme, dass die Zustandsdichten der Spitze und der Probe in der betrachteten Energiezone
annähernd konstant sind, ist der gemessene Tunnelstrom proportional zur Transmission und zur Spannung \(V\). Daraus 
folgt die zentrale Auswertungsrelation für das STM,


\begin{equation*}
I(d)\propto V\,\rho_\text{t}(E_F)\,\rho_\text{s}(E_F)\,e^{-2\kappa d},
\end{equation*}

wobei \(\rho_\text{t}(E_F)\) und \(\rho_\text{s}(E_F)\) die Zustandsdichten (Density of States, DOS) an der Fermienergie der Spitze bzw. Probe bezeichnen. 
Für typische Metallspitzen gilt \(E\ll\Phi\) und es wird \(\kappa\approx\sqrt{2m\Phi}/\hbar\) gesetzt, sodass häufig 

\begin{equation*}
I(d)\propto e^{-2\kappa d},\qquad \kappa\approx\frac{\sqrt{2m\Phi}}{\hbar}
\end{equation*}
verwendet wird \cite{schroedinger}.

\subsection{Tunnelstrom und lokale Zustandsdichte}
Der gemessene Tunnelstrom enthält neben Topographieinformation auch direkte elektronische Information der Probe, da er aus den energetischen
Übergängen zwischen besetzten Zuständen der einen Elektrode und unbesetzten Zuständen der anderen entsteht und durch die Tunneltransmission gewichtet
wird. Für kohärentes, elastisches Tunneln lässt sich der Gleichstrom bei angelegter Spannung \(V\) allgemein als Energieintegral über die Produktionen der 
Zustandsdichten und der Transmission schreiben,

\begin{equation*}
I(V)\propto\int_{-\infty}^{\infty}\rho_\text{t}(E-eV)\,\rho_\text{s}(E)\,T(E,d)\,\big[f(E-eV)-f(E)\big]\,\mathrm{d}E,
\end{equation*}


wobei \(\rho_\text{t}\) und \(rho_\text{s}\) die elektronischen Zustandsdichten von Spitze und Probe, \(T(E,d)\) die energie‑ und abstandsabhängige Tunneltransmission
und \(f(E)\) die Fermi‑Verteilungsfunktion sind. Unter den in STM‑Experimenten üblichen Bedingungen wie geringe Temperatur, kleine Biasspannungen, 
und eine Spitze mit annähernd konstanter DOS im betrachteten Energiebereich, ist die Besetzungsdifferenz auf ein schmales Energiefenster um die 
Fermi‑Energie beschränkt. Dann reduziert sich das Integral im führenden Näherungsgrad auf das integrierte Spektrum der Probenzustandsdichte
im Fenster zwischen \(E_\text{F}\) und \(E_\text{F}+eV\). In der Tersoff–Hamann‑Nähe, bei der die Spitze als s‑symmetrisches, energetisch flaches Reservoir
modelliert wird und das Tunnelmatrixelement proportional zur Probenwellenfunktion an der Spitzenposition \(\mathbf{r}_0\) ist, folgt die gebräuchliche Näherung

\begin{equation*}
I(V,\mathbf{r}_0)\propto\int_{E_\text{F}}^{E_\text{F}+eV}\rho_\text{s}(\mathbf{r}_0,E)\,\mathrm{d}E,
\end{equation*}

woraus durch Differentiation nach \(V\) die zentrale Relation der spektroskopischen Auswertung resultiert,

\begin{equation*}
\frac{\mathrm{d}I}{\mathrm{d}V}(V,\mathbf{r}_0)\propto\rho_s(\mathbf{r}_0,E_\text{F}+eV).
\end{equation*}


Damit stellt \(\mathrm{d}I/\mathrm{d}V\) unter den genannten Näherungen ein direktes Maß für die lokale Zustandsdichte der Probe bei der entsprechenden
Energie dar und bildet die Grundlage der Scanning Tunneling Spectroscopy \cite{carbillet}. 


\subsection{Piezoröhren}
Piezoröhren sind zylindrische Aktuatoren aus piezoelektrischer Keramik, deren Innen- und Außenflächen mit segmentierten Elektroden metallisiert sind.
Durch Polung der ferroelektrischen Domänen während der Herstellung wird eine reversible elektromechanische Kopplung etabliert. Der Aufbau einer Piezoröhre ist in \autoref{fig:piezo} dargestellt.
Anlegen einer elektrischen  Spannung an die Elektroden erzeugt eine lokale Deformation der Rohrwand, die in geeigneter Kombination axiale Längenänderungen sowie Biege‑ und Torsionsanteile 
(X,Y‑Richtungen) hervorruft. Im Rastertunnelmikroskop dient die Piezoröhre als Feinantrieb für die genaue Positionierung von Spitze bzw. Probe. Typische maximale
Ausschläge liegen im Bereich weniger Nanometer aufgrund der exponentiellen Abhängigkeit des Tunnelstroms vom Abstand.

\begin{figure}[H]
    \centering
    \includegraphics[width=0.4\textwidth]{bildertheorie/piezo.jpg}
    \caption{Aufbau und Funktionsweise des piezoelektrischen Rohraktuators \cite{piezo}.}
    \label{fig:piezo}
  \end{figure}


Für kleine elektrische Felder lässt sich die axiale Längenänderung in linearer Näherung als


\begin{equation*}
\Delta L \approx d_{33}\,L_0\,E = d_{33}\,L_0\,\frac{V}{t},
\end{equation*}

schreiben, wobei \(d_{33}\) der piezoelektrische Koeffizient, \(L_0\) die nominale Länge, \(t\) die Wandstärke und \(V\) die angelegte Spannung ist.
Diese lineare Relation ist jedoch nur eine erste Näherung, für reale Piezoröhren sind mehrere nichtlineare und zeitabhängige Phänomene von Bedeutung, 
die die Qualität von STM‑Messungen direkt beeinflussen.
Fünf zentrale Effekte sind experimentell relevant: Hysterese, Creep, Nichtlinearität/Skalenabweichung, Kopplung zwischen Achsen (Cross‑Talk) und mechanische Resonanzen.
Hysterese führt zu unterschiedlicher Verschiebung bei auf- und ablaufender Spannung und erzeugt dadurch systematische Positionsfehler, 
welche sich in Rasterverzerrungen und Asymmetrien äußern. Creep bezeichnet die zeitabhängige Nachbewegung nach einem Spannungssprung und verursacht langsamen Drift
im Scanbild. Dieser Effekt ist besonders kritisch bei langsamen Linienraten oder beim Abtasten feiner Strukturen. Nichtlinearität und veränderliche Skalenfaktoren bedingen
Volt‑Abweichungen über den Arbeitsbereich und führen zu Maßstabsfehlern. Cross‑Talk entsteht, weil eine Ansteuerung eines Segments parasitäre Bewegungen 
in den anderen Achsen induziert, ohne Korrektur erscheinen Rotations‑ oder Scherartefakte. Mechanische Eigenmoden von Röhre und Montage können bei bestimmten
Anregungsfrequenzen resonant angeregt werden und sind Ursache für periodische Artefakte bei zu hohen Scanraten.
Diese Effekte lassen sich nicht vollständig eliminieren, aber teilweise kompensieren. Geschlossene Regelkreise mit kapazitiven oder interferometrischen Wegsensoren 
ermöglichen echtes Closed‑Loop‑Feedback und reduzieren Hysterese und Creep erheblich, sofern Sensorauflösung und Bandbreite ausreichend sind. Fehlt eine direkte Wegmessung,
so kann eine regelmäßige Kalibrierungen an Referenzgittern helfen, verlässlichere Messwerte zu erzielen.
Vor Messreihen sind Kalibrierungen der Volt‑Skalierung und der Achsentopologie durchzuführen. Aufnahmen in Vorwärts‑ und Rückwärtsrichtung liefern Hinweise auf Hystereseeffekte 
und erlauben Korrekturen, langsame Linienraten und ausreichende thermische Stabilisierung reduzieren Creep‑ und Driftfehler. FFT‑gestützte Auswertungsschritte sind vorzuziehen,
da periodische Strukturen so präziser im Frequenzraum bestimmt und systematische Verzerrungen identifiziert werden können. Ebenso sind bei der Bestimmung von Sprunghöhen auf
Gold die Höhenkalibrierung und die Eliminierung von Spitzenartefakten essenziell, da die lateralen und vertikalen Sensitivitäten der Piezoröhre sowie mögliche Cross‑Talk‑Beiträge
direkten Einfluss auf die gemessenen Werte haben.
Als technische Alternativen zu Piezoröhren kommen piezostack‑Aktuatoren, bimorphe Biegebalken sowie Kombinationen aus Grobverstellern (Inchworm bzw. stick‑slip‑Mechanismen) und
feinen Piezoaktoren in Betracht \cite{lueth} \cite{piezo1}. 

Die Piezoröhren werden über sogenannte PID-Regler gesteuert. Ein PID-Regler besteht aus drei Anteilen: proportional (P), integral (I) und differential (D) — und übersetzt die Abweichung des gemessenen Tunnelstroms vom Sollwert in eine Stellgröße 
für die $Z$‑Position des Piezoaktors. Formal lässt sich die Stellgröße in Zeitdomäne schreiben als

\begin{equation}
u(t)=K_P\,e(t) + K_I\int_0^t e(\tau)\,\mathrm{d}\tau + K_D\frac{\mathrm{d}e(t)}{\mathrm{d}t},
\end{equation}

wobei \(e(t)\) der Regelabweichung entspricht und \(K_P\), \(K_I\), \(K_D\) die Verstärkungsfaktoren der jeweiligen Anteile sind. Physikalisch wird diese Stellspannung über den 
piezoelektrischen Effekt in eine Längenänderung umgesetzt. Die vom Regler erzeugte elektrische Ansteuerung \(u(t)\) ist damit die Eingangsgröße für das mechanische Subsystem des Scanners.
Die mechanische Dynamik des durch die Piezoansteuerung bewegten Teils lässt sich in guter Näherung durch ein gedämpftes Massen‑Feder‑System beschreiben. 
Für die effektive Koordinate \(x(t)\) des bewegten Teils gilt dann

\begin{equation}
m\ddot{x} + c\dot{x} + kx = F_{\mathrm{drive}}(t),
\end{equation}

wobei \(m\) die effektive Masse, \(c\) die Dämpfung, \(k\) die effektive Steifigkeit der Baugruppe und \(F_{\mathrm{drive}}(t)\) die durch das Piezoelement erzeugte Antriebskraft darstellt. 
Die Stellspannung \(u(t)\) ist über die Piezocharakteristik proportional zur erzeugten Deformation und somit indirekt mit \(F_{\mathrm{drive}}(t)\) verknüpft. Nichtlineare Effekte wie Piezo‑Hysterese 
und Creep sowie Kopplungen an statische Vorspannung und Kontaktbedingungen modifizieren diese Beziehung.
Physikalisch wirkt der proportionale Anteil \(K_P e(t)\) als unmittelbares Stellsignal proportional zur aktuellen Regelabweichung, in mechanischer Sprache bedeutet das eine schnelle Änderung von 
\(F_{\mathrm{drive}}(t)\) und damit erhöhte Beschleunigungsanforderungen an die Masse \(m\). Große Beschleunigungsanteile erzeugen hohe Trägheitskräfte \(m\ddot{x}\), können den Übergang von Haftung
zu Gleiten in piezo‑inertialen Kontakten begünstigen und mechanische Eigenmoden anregen. Der integrale Anteil \(K_I\!\int_0^t e(\tau)\,\mathrm{d}\tau\) akkumuliert langfristige Fehler und verschiebt 
so die mittlere Stellposition, wodurch langsame systematische Abweichungen wie Piezo‑Creep oder thermisch bedingte Drift ausgeglichen werden. Diese sukzessive Anpassung entspricht einer langsamen Änderung 
des Mittelwerts von \(F_{\mathrm{drive}}(t)\) und somit von \(x(t)\). Da die Kontaktkräfte (statische Haftkraft \(F_{\mathrm{stick}}\), kinetische Reibung \(F_{\mathrm{kin}}\)) in der Praxis variabel sind,
kann das Integral jedoch falsche Korrekturen aufsummieren.
Der differentielle Anteil \(K_D\,\mathrm{d}e/\mathrm{d}t\) reagiert auf die Änderungsrate des Fehlers und wirkt dämpfend auf schnelle Störungen. Physikalisch entspricht dies einer zusätzlichen
viskosen Dämpfung der schnellen Dynamik und reduziert Überschwingen, verstärkt jedoch hochfrequentes Messrauschen, sofern vor der Differentiation keine geeignete Filterung erfolgt.
Beispielhafte Signalaufnahmen eines PID-Reglers sind in \autoref{fig:pid} zu sehen \cite{pidregler}.
\begin{figure}[H]
    \centering
    \includegraphics[width=0.5\textwidth]{bildertheorie/pid.jpg}
    \caption{Vergleich der synchron aufgezeichneten Signale eines PID-Reglers \cite{pid}.}
    \label{fig:pid}
\end{figure}


\subsection{Proben}
\subsubsection{HOPG}
Bei Highly Ordered Pyrolytic Graphite, kurz HOPG, handelt es sich um ein Material, welches aus mehreren Graphenschichten, zu sehen in \autoref{fig:graphen}, besteht,
die in einer $a$-$b$-Struktur zu einander stehen, dargestellt in \autoref{fig:hopg}

\begin{figure}[H]
    \centering
    \includegraphics[width=0.4\textwidth]{bildertheorie/Bild.jpg}
    \caption{Schematische Darstellung der Gitterstruktur von Graphen \cite{Graphen}.}
    \label{fig:graphen}
  \end{figure}
  
\begin{figure}[H]
    \centering
    \includegraphics[width=\textwidth]{bildertheorie/hopg.jpg}
    \caption{Schematischer Aufbau von HOPG. Der weiter oben liegende Layer ist schwarz, der darunter liegende Layer weiß. Aufgrund der $a$-$b$-Struktur der Schichten ist je ein Atom der zweiatomigen Basis direkt über einem anderen Atom einer darunter liegenden Basis \cite{hopg}.}
    \label{fig:hopg}
\end{figure}

Graphen ist eine einzelne Schicht des Graphitgitters und bildet ein zweidimensionales, hexagonales Kohlenstoffnetzwerk. Jedes Kohlenstoffatom ist sp\textsuperscript{2}-hybridisiert 
und bindet kovalent an drei Nachbarn in der Ebene. Die verbleibenden p\textsubscript{z}-Orbitale bilden delokalisierte \(\pi\)-Zustände senkrecht zur Ebene, die die charakteristische 
elektronische Struktur erzeugen. Die in‑plane Gitterkonstante beträgt \(r \approx \SI{0.246}{\nano\meter}\), der nächste Nachbarabstand \(r_{\mathrm{CC}}\approx 0{.}142\ \mathrm{nm}\).
Aufgrund der Wechselwirkung übereinander liegender Kohlenstoffatome ändern sich die Zustandsdichten eben jener Atome, sodass unter einem Rastertunnelmikroskop
nur jedes zweite Atom gemessen werden kann. Deshalb wird in diesem Versuch der Abstand $r \approx \SI{0.246}{\nano\meter}$ vermessen.

Aufgrund der periodischen Struktur liegt es nahe, die reale Struktur über eine Fouriertransformation in den reziproken Raum zu übersetzen, schematisch dargestellt in \autoref{fig:ft}.

\begin{figure}[H]
    \centering
    \includegraphics[width=\textwidth]{bildertheorie/ft.png}
    \caption{Sechs Bilder mit periodischen Mustern verschiedener Komplexität und ihre Übersetzung über eine Fouriertransformation in den reziproken Raum. Periodische Strukturen
    werden in Punkte übersetzt, die Periodizität in eine bestimmte Richtung darstellen \cite{ft}.}
    \label{fig:ft}
\end{figure}

In dieser Darstellung werden periodische Strukturen im realen Raum zu Punktpaaren im reziproken Raum, die vom Ursprung aus in Richtung der Periodizität liegen. Ihr abstand enthält Informationen über
die Frequenz der Periodizität. Diese Abstände im reziproken Raum und realen Raum stehen invers zu einander, was über die Definition des reziproken Gitters

\begin{equation*}
  b_i^* = \frac{2\pi}{V_\text{E}} (a_j \times a_k)
\end{equation*}

nachgerechnet werden kann. $b_i^*$ ist ein reziproker Gittervektor, $a_j$ und $a_k$ sind reale Gittervektoren. Auch scheinbar komplexe Strukturen können so in wenige Punkte zerlegt werden.
Für die Abstände im hexagonalen Gitter ergibt sich
\begin{equation*}
  \left|r\right| = \frac{4pi}{\sqrt{3}k}
\end{equation*}
für $k$ in Radians pro Länge. Für $k$ in Zyklen pro Länge wird ein Faktor von $2\pi$ eliminiert,
\begin{equation}
    \left|r\right| = \frac{2}{\sqrt{3}k}.
    \label{eq:rez_to_real}
\end{equation}
\cite{ft}
HOPG liefert aufgrund der gut orientierten Graphenschichten und der ausgeprägten \(\pi\)-Zustände einen starken, periodischen elektronischen Kontrast. 
Die atomare Periodizität und die hohe laterale Homogenität führen zu scharfen, gut zuordenbaren Peaks in der 2D‑Fourier‑Transformation, weshalb HOPG als 
Lateralkalibrierstandard und Referenzgitter weit verbreitet ist. Frisch aufgebrochene HOPG‑Flächen sind leicht reproduzierbar, was die experimentelle Vergleichbarkeit 
verbessert. Einschränkungen ergeben sich durch Sublattice‑abhängige Kontraste, sowie lokale Defekte, die lokale Abweichungen verursachen.





\subsubsection{Gold}

Gold kristallisiert im kubisch‑flächenzentrierten (fcc) Gitter, dargestellt in \autoref{fig:fcc}, mit der Gitterkonstante $a_0\approx \SI{0.408}{\nano\meter}$.

\begin{figure}[H]
    \centering
    \includegraphics[width=0.4\textwidth]{bildertheorie/fcc.jpeg}
    \caption{Aufbau eines fcc-Gitters \cite{fcc}.}
    \label{fig:fcc}
  \end{figure}

Für die (111)-Ebene ergeben sich die charakteristischen Längenskalen

\begin{equation*}
a_{\mathrm{NN}}=\frac{a_0}{\sqrt{2}}\approx 0{.}288\ \mathrm{nm},\qquad
h_{111}=\frac{a_0}{\sqrt{3}}\approx 0{.}236\ \mathrm{nm}.
\end{equation*}


Elektronisch dominieren bei Gold delokalisierte s‑ und d‑Elektronen. Oberflächennahe Zustände und Rekonstruktionen modifizieren die ideale Oberflächenstruktur. 
Gold eignet sich besonders zur Untersuchung von monoatomaren Stufen und Oberflächenrekonstruktionen, da die Stufenhöhen klar definierte Vertikalskalen liefern.
Die atomaren topographischen Amplituden sind jedoch klein (Sub‑Å‑Bereich) und der laterale atomare Kontrast in oberflächenempfindlichen Techniken wie STM hängt 
stark von Spitzengeometrie, Bias und elektronischer Struktur ab. Rekonstruktive Superstrukturen überlagern den reinen Gitterkontrast und erschweren die direkte 
Bestimmung der atomaren Gitterkonstante, daher sind höhere Anforderungen an Probenvorbereitung, Spitzenzustand und Auswertemethoden nötig. 







