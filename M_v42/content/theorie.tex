\section{Theorie}
\label{sec:Theorie}
\subsection{Tunneleffekt}
Der Tunneleffekt ist ein rein quantenmechanisches Phänomen, bei dem Teilchen eine klassisch unüberwindbare Potentialbarriere mit endlicher Wahrscheinlichkeit durchdringen.
Im Rastertunnelmikroskop führt dies dazu, dass Elektronen zwischen einer metallischen Spitze und der Probenoberfläche durch die Vakuumbarriere tunneln, obwohl ihre Energie
unter der effektiven Barrierenhöhe liegt. Da die Tunnelwahrscheinlichkeit stark vom Abstand zwischen Spitze und Probe abhängt, lässt sich über die Messung des Tunnelstroms 
die Topographie der Oberfläche mit atomarer Auflösung rekonstruieren.
Zur quantitativen Beschreibung betrachtet man ein Elektron mit Energie \(E\) in einer eindimensionalen Barriereregion mit konstanter effektiver Barrierenhöhe \(\Phi\) über 
der Fermienergie und Breite \(d\). Die zeitunabhängige Schrödingergleichung in der Barriereregion lautet

\begin{equation}
-\frac{\hbar^2}{2m}\frac{\mathrm{d}^2\psi}{\mathrm{d}x^2}+(\Phi-E)\psi=0.
\end{equation}

Für \(E<\Phi\) ergibt sich ein reeller Dämpfungsparameter

\begin{equation}
\kappa=\frac{\sqrt{2m(\Phi-E)}}{\hbar},
\end{equation}

wodurch die Wellenfunktion in der Barriereregion exponentiell abklingt, \(\psi(x)\propto e^{-\kappa x}\).
Die Transmissionswahrscheinlichkeit durch die Barriere skaliert in führender Näherung wie


\[
T\propto e^{-2\kappa d}.
\]

Bei kleinen Biasspannungen und unter der Annahme, dass die Zustandsdichten der Spitze und der Probe in der betrachteten Energiezone
annähernd konstant sind, ist der gemessene Tunnelstrom proportional zur Transmission und zur Spannung \(V\). Daraus 
folgt die zentrale Auswertungsrelation für das STM:


\begin{equation}
I(d)\propto V\,\rho_t(E_F)\,\rho_s(E_F)\,e^{-2\kappa d},
\end{equation}

wobei \(\rho_t(E_F)\) und \(\rho_s(E_F)\) die Zustandsdichten an der Fermienergie der Spitze bzw. Probe bezeichnen. 
Für typische Metallspitzen gilt \(E\ll\Phi\) und man setzt \(\kappa\approx\sqrt{2m\Phi}/\hbar\), sodass häufig verwendet wird

\begin{equation}
I(d)\propto e^{-2\kappa d},\qquad \kappa\approx\frac{\sqrt{2m\Phi}}{\hbar}.
\end{equation}

\subsection{Tunnelstrom und lokale Zustandsdichte}
Der gemessene Tunnelstrom enthält neben Topographieinformation auch direkte elektronische Information der Probe, da er aus den energetischen
Übergängen zwischen besetzten Zuständen der einen Elektrode und unbesetzten Zuständen der anderen entsteht und durch die Tunneltransmission gewichtet
wird. Für kohärentes, elastisches Tunneln lässt sich der Gleichstrom bei angelegter Spannung \(V\) allgemein als Energieintegral über die Produktionen der 
Zustandsdichten und der Transmission schreiben.

\begin{equation}
I(V)\propto\int_{-\infty}^{\infty}\rho_t(E-eV)\,\rho_s(E)\,T(E,d)\,\big[f(E-eV)-f(E)\big]\,\mathrm{d}E,
\end{equation}


wobei \(\rho_t\) und \(\rho_s\) die elektronischen Zustandsdichten von Spitze und Probe, \(T(E,d)\) die energie‑ und abstandsabhängige Tunneltransmission
und \(f(E)\) die Fermi‑Verteilungsfunktion sind. Unter den in STM‑Experimenten üblichen Bedingungen wie geringe Temperatur, kleine Biasspannungen, 
und eine Spitze mit annähernd konstanter DOS im betrachteten Energiebereich, ist die Besetzungsdifferenz auf ein schmales Energiefenster um die 
Fermi‑energie beschränkt. Dann reduziert sich das Integral im führenden Näherungsgrad auf das integrierte Spektrum der Probenzustandsdichte
im Fenster zwischen \(E_F\) und \(E_F+eV\). In der Tersoff–Hamann‑Nähe, bei der die Spitze als s‑symmetrisches, energetisch flaches Reservoir
modelliert wird und das Tunnelmatrixelement proportional zur Probenwellenfunktion an der Spitzenposition \(\mathbf{r}_0\) ist, folgt die gebräuchliche Näherung

\begin{equation}
I(V,\mathbf{r}_0)\propto\int_{E_F}^{E_F+eV}\rho_s(\mathbf{r}_0,E)\,\mathrm{d}E,
\end{equation}

woraus durch Differentiation nach \(V\) die zentrale Relation der spektroskopischen Auswertung resultiert.

\begin{equation}
\frac{\mathrm{d}I}{\mathrm{d}V}(V,\mathbf{r}_0)\propto\rho_s(\mathbf{r}_0,E_F+eV).
\end{equation}


Damit stellt \(\mathrm{d}I/\mathrm{d}V\) unter den genannten Näherungen ein direktes Maß für die lokale Zustandsdichte der Probe bei der entsprechenden
Energie dar und bildet die Grundlage der Scanning Tunneling Spectroscopy. 

\subsection*{Exponentielle Zerfallsrate \(\kappa\)}

Die exponentielle Zerfallsrate \(\kappa\) charakterisiert den räumlichen Abfall der elektronischen Wellenfunktion innerhalb der Vakuumbarriere und verbindet die 
effektive Barrierenhöhe mit der Abstandabhängigkeit der Tunnelwahrscheinlichkeit. Betrachtet man ein eindimensionales Potential mit konstanter Barrierenhöhe \(\Phi\)
oberhalb der Elektronenenergie \(E\) (für \(E<\Phi\)), lautet die zeitunabhängige Schrödingergleichung in der Barriereregion

\begin{equation}
-\frac{\hbar^2}{2m}\frac{\mathrm{d}^2\psi}{\mathrm{d}x^2}+(\Phi-E)\psi=0.
\end{equation}

Für konstantes \(\Phi\) führt dies zu Lösungen exponentiellen Typs. Mit der Definition des Dämpfungsparameters

\begin{equation}
\kappa=\frac{\sqrt{2m(\Phi-E)}}{\hbar}
\end{equation}

fällt die Wellenfunktion in der Barriereregion wie \(\psi(x)\propto e^{-\kappa x}\) ab. Die Transmissionswahrscheinlichkeit \(T\) durch eine Barrierendicke 
\(d\) skaliert in führender Näherung mit dem Quadrat der Amplitude am Barriereende, sodass gilt

\begin{equation}
T\propto e^{-2\kappa d}.
\end{equation}

Im STM‑Kontext ist der Tunnelstrom proportional zur Transmission, woraus die zentrale Abstand‑Strom‑Relation folgt.

\begin{equation}
I(d)\propto e^{-2\kappa d}.
\end{equation}

Für Energien nahe dem Ferminiveau wird häufig \(E\ll\Phi\) angenommen, womit sich die gebräuchliche Näherung

\begin{equation}
\kappa\approx\frac{\sqrt{2m\Phi}}{\hbar}
\end{equation}

ergibt. Umgekehrt lässt sich aus einem bestimmten experimentellen \(\kappa\) die effektive Barrierenhöhe berechnen durch

\begin{equation}
\Phi=\frac{(\hbar\kappa)^2}{2m}.
\end{equation}

Damit verbindet \(\kappa\) die quantenmechanische Beschreibung der Wellenfunktion in der Barriereregion direkt mit der
makroskopisch beobachtbaren exponentiellen Abhängigkeit des Tunnelstroms vom Spitzen‑Probendistanz \(d\).
