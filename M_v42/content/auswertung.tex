\section{Auswertung}
\label{sec:Auswertung}

Für die Auswertung wird die \texttt{Python}-Bibliothek \texttt{numpy} \cite{numpy} benutzt. Mögliche Fits entstehen mit \texttt{curve\_fit} aus \texttt{scipy.optimize}. \cite{scipy}.
Die Fehlerrechnung wird mit \texttt{uncertainties} \cite{uncertainties} durchgeführt. Plots entstehen mit \texttt{matplotlib.pyplot} \cite{matplotlib}.
Bildbearbeitungen für das HOPG werden mit \texttt{skimage} durchgeführt. Die Auswertung der Goldprobe wird mit \texttt{Gwyddion} durchgeführt.

\subsection{HOPG}

Um Rauschartefakte zu unterdrücken, werden die Messungen der HOPG-Probe mit einem Gauß-Filter versehen.
Anschließend werden die Intensitäten fouriertransformiert. Die Abstände benachbarter Peaks im reziproken Raum entsprechen den reziproken Gittervektoren zwischen den hellen Spots im Realraum.
Über \autoref{eq:rez_to_real} lassen sich somit die Gittervektoren bzw. die Abstände heller Spots berechnen.





\subsection{Gold}
Für Gold können keine atomar aufgelösten Bilder gemacht werden, aber aus den Messungen mit dem Mikroskop lassen sich Informationen über Schichtdicken gewinnen.
Mit \texttt{Gwyddion} können die Messungen zunächst nachbearbeitet werden. Dafür werden sie zuerst geebnet und anschließend das Minimum der Daten auf null gesetzt. 
Für die Auswertung der Höhe wird ein vom Programm bereitgestelltes Tool benutzt, das die Höhenverteilung in ausgewählten Bereichen darstellt. Dafür wird eine 35 mal 35 Pixel große Fläche
händisch über das Scanbild gefahren und in Echtzeit die statistische Verteilung der Höhen beobachtet. Es können optisch bereits Stellen ausgemacht werden, an denen
Höhenunterschiede sichtbar sind. Wenn die Statistik mindestens zwei eindeutige Peaks zeigt, werden diese markiert und die Differenz der entsprechenden Höhen ausgerechnet.
Dies wird für den ersten Vorwärtsscan drei Mal wiederholt. Für den zweiten Vorwärtsscan sowie den zugehörigen Rückwärtsscan wird dies je ein Mal durchgeführt.

