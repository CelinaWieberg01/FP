\section{Diskussion}
\label{sec:Diskussion}
Bei der Durchführung unserer Messungen und Analysen gibt es mehrere kritische Aspekte, 
die berücksichtigt werden müssen, um die Genauigkeit und Zuverlässigkeit der Ergebnisse sicherzustellen.
Zunächst könnte die Wahl der Fehler genauer sein. Es ist entscheidend, dass die Unsicherheiten
in den Messungen korrekt bestimmt und berücksichtigt werden, um die Präzision der Ergebnisse zu gewährleisten.
Während der Messungen des Stromes wurde zu Beginn häufig umgepolt. Dies kann zu systematischen Fehlern führen 
und sollte in zukünftigen Experimenten minimiert oder vermieden werden.
Ein weiteres wichtiges Detail ist die Heizrate, die über die Zeit nicht konstant blieb, sondern eher einer logistischen Kurve folgte.
Dies deutet darauf hin, dass die Temperaturregelung nicht gleichmäßig war und Anpassungen an der Steuerung vorgenommen werden sollten, 
um eine gleichmäßigere Heizrate zu gewährleisten. Die Annahme eines linearen Hintergrunds für die Datenanalyse ist ebenfalls
kritisch zu betrachten. Möglicherweise wäre die Verwendung eines exponentiellen Hintergrunds sinnvoller, um die Daten genauer zu 
modellieren und zu analysieren. Bei der numerischen Integration könnten alternative Integrationsmethoden gewählt werden, um die
Genauigkeit der Berechnungen zu verbessern. Dies könnte dazu beitragen, systematische Fehler zu minimieren und präzisere Ergebnisse zu erzielen.
Die experimentell bestimmten Heizraten wichen von den theoretischen Werten ab. Die über den Fit bestimmten Heizraten waren:
\begin{equation}
b_1 =  \qty{1.24(1)}{\kelvin\per\minute}
\end{equation}
\begin{equation}
b_2 = \qty{1.67(2)}{\kelvin\per\minute}
\end{equation}

Die theoretischen Werte sollten $\qty{1.5}{\kelvin\per\minute}$ und $\qty{2.0}{\kelvin\per\minute}$ betragen.
Die prozentuale Abweichungen sind
\begin{equation}
    \Delta W_1= \frac{|1.24 - 1.5|}{1.5} \cdot100 \approx17.33\%
\end{equation}
für die erste Heizrate und
\begin{equation}
    \Delta W_2= \frac{|1.67 - 2.0|}{2.0} \cdot100 \approx16.5\%
\end{equation}
für die zweite. 
Diese Abweichungen sind signifikant und deuten darauf hin, dass es systematische Unterschiede zwischen den experimentellen 
Bedingungen und den theoretischen Annahmen gibt. Es ist wichtig, diese Diskrepanzen weiter zu untersuchen und mögliche 
Ursachen wie experimentelle Fehler, Abweichungen in den Materialien oder Messgeräteungenauigkeiten zu identifizieren und zu korrigieren.
Durch die Beachtung dieser Aspekte und die Implementierung geeigneter Korrekturmaßnahmen können wir die Genauigkeit und
Zuverlässigkeit der experimentellen Ergebnisse verbessern.