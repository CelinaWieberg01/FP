\section{Zielsetzung}
\label{sec:Zielsetzung}
Der vorliegende Versuch dient dazu, die Wechselwirkungen zwischen \(\alpha\)-Teilchen und Atomkernen experimentell zu 
untersuchen und damit das Rutherford‑Streuungsmodell zu prüfen. An dünnen Metallfolien, insbesondere Gold, werden die 
Winkelverteilung der gestreuten Teilchen sowie die Abhängigkeit des Streuverhaltens von der Kernladungszahl \(Z\) verschiedener 
Targetmaterialien ermittelt. Ergänzend wird die Folienstärke über die Messung des Energieverlusts der \(\alpha\)-Teilchen bestimmt, 
wodurch material- und aufbauabhängige Effekte quantifiziert werden können. Durch Messungen an Folien unterschiedlicher Dicke
lässt sich zudem der Einfluss von Mehrfachstreuung abschätzen und die Gültigkeitsgrenze einfacher Streumodelle aufzeigen. 
