\section{Theorie}
\label{sec:Theorie}
\subsection*{Energieverlust und Wechselwirkung mit Hüllenelektronen}
Der kontinuierliche Energieverlust geladener Teilchen durch Anregung und Ionisation der Hüllenelektronen wird
durch das Störungsverhalten der Elektronen beschrieben und für schnelle, nichtrelativistische bis moderat relativistische 
Teilchen durch die Bethe‑Bloch‑Gleichung genähert.


\begin{equation}
\frac{\mathrm{d}E}{\mathrm{d}x}
= -\frac{4\pi e^4 z^2 N Z}{m_0 v^2 (4\pi\varepsilon_0)^2}\,\ln\!\left(\frac{2 m_0 v^2}{I}\right),
\end{equation}


wobei \(z\) die Ladungszahl des Projektils, \(N\) die Atomdichte des Targets, \(Z\) die Kernladungszahl, \(m_0\) 
die Elektronenruhemasse, \(v\) die Teilchengeschwindigkeit und \(I\) die mittlere Ionisationsenergie des Materials bezeichnet.
Aus dem Energieverlust folgt durch Integration über die Energie die Reichweite \(R(E)\) der Teilchen, sodass die effektive 
Folienstärke \(x\) aus dem gemessenen Energieverlust \(\Delta E\) über

\begin{equation}
x \;=\; \int_{E_\alpha-\Delta E}^{E_\alpha} \left(\frac{\mathrm{d}E}{\mathrm{d}x}\right)^{-1}\mathrm{d}E
\end{equation}


bestimmt werden kann. Die Atomdichte \(N\) wird durch die Materialdichte \(\rho\), die molare Masse \(A\) und die Avogadrozahl \(N_A\) über

\begin{equation}
N \;=\; \frac{\rho N_A}{A}
\end{equation}

ausgedrückt. Für die Kalibrierung der Pulshöhen ist zu berücksichtigen, dass die gemessene Pulshöhe proportional zur verbliebenen kinetischen Energie der \(\alpha\)-Teilchen ist.

\subsection{Elastische Coulomb‑Streuung und differentieller Wirkungsquerschnitt}
Die Ablenkung der \(\alpha\)-Teilchen durch die Coulombkraft des positiv geladenen Kerns wird durch den differentiellen Wirkungsquerschnitt beschrieben. 
Unter der Annahme der Einzelstreuung und der Vernachlässigung quantenmechanischer Feinheiten ergibt sich für den Rutherford‑Wirkungsquerschnitt 
die Form

\begin{equation}
\frac{\mathrm{d}\sigma}{\mathrm{d}\Omega}(\theta)
= \left(\frac{z Z e^2}{16\pi\varepsilon_0 E_\alpha}\right)^{\!2}\frac{1}{\sin^4(\theta/2)},
\end{equation}


wobei \(E_\alpha\) die kinetische Energie der \(\alpha\)-Teilchen und \(\theta\) der Streuwinkel ist. Der differentielle Wirkungsquerschnitt 
gibt die Wahrscheinlichkeit an, in einen bestimmten Raumwinkel \(\mathrm{d}\Omega\) gestreut zu werden, und skaliert quadratisch mit dem Produkt
der Ladungszahlen \(zZ\). Experimentell wird die Zählrate \(I(\theta)\) gemessen, die mit dem differentiellen Wirkungsquerschnitt durch die Beziehung

\begin{equation}
I(\theta) \;=\; I_0\, n x\, \frac{\mathrm{d}\sigma}{\mathrm{d}\Omega}(\theta)\,\Delta\Omega\,\varepsilon
\end{equation}

verknüpft ist. Hierbei bezeichnet \(I_0\) den einfallenden Teilchenstrom, \(n\) die Zahl der Streuzentren pro Volumen, \(x\) die Folienstärke, 
\(\Delta\Omega\) den vom Detektor abgedeckten Raumwinkel und \(\varepsilon\) die Detektionseffizienz. Für kleine Detektoren in ausreichendem 
Abstand \(r\) zur Streustelle kann der Raumwinkel näherungsweise durch

\begin{equation}
\Delta\Omega \approx \frac{A\cos\alpha}{r^2}
\end{equation}

beschrieben werden, wobei \(A\) die effektive Detektorfläche und \(\alpha\) der Winkel zwischen Detektorebene und Sichtlinie ist.

\subsection{Näherungen, Abschirmung und endliche Kernstruktur}
Bei der Anwendung der Rutherford‑Formel werden mehrere Näherungen getroffen, die experimentell überprüft werden müssen. Die Einzelstreuungsannahme 
ist nur für dünne Folien gültig. Bei zunehmender Folienstärke werden Mehrfachstreuungen relevant und führen zu einer Aufweitung der Winkelverteilung. 
Bei großen Impactparametern wird die Kernladung durch die Hüllenelektronen abgeschirmt, sodass bei sehr kleinen Streuwinkeln Abweichungen von der
reinen Coulomb‑Form auftreten. Bei sehr kleinen Impactparametern wird die Annahme eines punktförmigen Kerns unzureichend, die endliche Ladungsverteilung 
des Kerns führt zu Abweichungen bei großen Streuwinkeln und ermöglicht Rückschlüsse auf den effektiven Kernradius.

\subsection{Mehrfachstreuung}
Die Streuung an vielen kleinen Stößen innerhalb der Folie führt zu einer statistischen Aufweitung des Streuwinkels, die durch Näherungsformeln beschrieben werden kann. 
Für die charakteristische RMS‑Ablenkung \(\theta_0\) wird häufig die Highland‑Formel verwendet,

\begin{equation}
\theta_0 \approx \frac{13{,}6\ \mathrm{MeV}}{\beta p c}\,z\,\sqrt{\frac{x}{X_0}}\left[1+0{,}038\ln\!\left(\frac{x}{X_0}\right)\right],
\end{equation}

wobei \(\beta\) und \(p\) die relative Geschwindigkeit bzw. der Impuls des Projektils, \(z\) dessen Ladungszahl, \(x\) die Materialdicke und \(X_0\) die Strahlungslänge
des Materials bezeichnen. Durch Vergleich der gemessenen Winkelverteilung für verschiedene Folienstärken kann der Übergang von der Einzelstreuung zur Mehrfachstreuung 
quantifiziert werden.

\cite{sample}
