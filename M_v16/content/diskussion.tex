\section{Diskussion}
\label{sec:Diskussion}

\subsection{Verstärkung}
Wie in \autoref{fig:oszilloskop} dargestellt ist der Verstärker unerlässlich für diesen Versuch.
Noch relevanter wäre er bei Messungen, die abhängig vom Umgebungsdruck bzw. der Dichte des Mediums sind,
wie z. B. die Messung des Energieverlustes der Teilchen.

\subsection{Strahlcharakteristik und Wirkungsquerschnitte}

\subsubsection{Vakuum}
Die Messung des Vakuums zeigt keine besonderen Auffälligkeiten. Wenn der Detektor mitten im Strahl steht, werden viele
Events registriert; bei kleiner Abweichung fallen diese rapide ab. Die hier gemessenen Ausreißer in \autoref{fig:vakuum} sind auf äußere Störungen zurückzuführen wie elektrische Störungen im Netz.
Über den Diskriminator wurde versucht, die Events rauszufiltern.

\subsubsection{Gold}
Die Messung der Streuung in der Goldfolie zeigt qualitativ den zu erwartenden Verlauf, jedoch mit einer erkennbaren Abweichung für kleine Winkel.
Die kleineren Winkel liegen in dem Winkelbereich, in dem 
in \autoref{fig:vakuum} die Zählrate vergleichsweise hoch, was auf eine Verzerrung der Messwerte hindeutet. Da diese Zählrate bei größeren Winkel sehr schnell
abnimmt, ist der Effekt dort weniger relevant.

Da Gold eine hohe Dichte von $\rho_\text{Au} = \qty{19320}{\kilo\gram\per\meter\cubed}$ bei ähnlicher molaren Masse zu Bismut besitzt, sind hier aufgrund der größeren Anzahl an Streuzentren höhere Streuquerschnitte bei größeren Winkeln zu erwarten,
was ein Vergleich der Messungen an beiden Folien bestätigt.

\subsubsection{Bismut}
Die Messung an der Bismutfolie ist ähnlich wie bei der Goldfolie. Hier nimmt die Differenz zwischen Theoriekurve und Messwerten jedoch schneller ab.

Die Dichte von Bismut ist mit $\rho_\text{Bi} = \qty{9790}{\kilo\gram\per\meter\cubed}$ fast halb so groß wie die von Gold, wodurch bei größeren Winkeln niedrigere Streuquerschnitte zu erwarten sind, was die Messung bestätigt.

