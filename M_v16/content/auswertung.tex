\section{Auswertung}
\label{sec:Auswertung}

Für die Auswertung wird die \texttt{Python}-Bibliothek \texttt{numpy} \cite{numpy} benutzt. Die Fits entstehen mit \texttt{curve\_fit} aus \texttt{scipy.optimize} \cite{scipy}.
Die Fehlerrechnung wird mit \texttt{uncertainties} \cite{uncertainties} durchgeführt. Plots entstehen mit \texttt{matplotlib.pyplot} \cite{matplotlib}.

\subsection{Verstärkung}
Um die Wirkung des Verstärkers zu demonstrieren, werden Oszilloskopbilder jeweils mit und ohne Verstärker aufgenommen, dargestellt in \autoref{fig:oszilloskop}

\begin{figure}[H]
    \centering
    \begin{subfigure}[b]{0.48\textwidth}
      \centering
      \includegraphics[width=\textwidth]{plots/unverstaerkt.JPG}
      \caption{Ohne Verstärker.}
      \label{fig:ohne_verstaerker}
    \end{subfigure}\hfill
    \begin{subfigure}[b]{0.48\textwidth}
      \centering
      \includegraphics[width=\textwidth]{plots/verstaerkt.JPG}
      \caption{Mit Verstärker.}
      \label{fig:mit_verstaerker}
    \end{subfigure}
    \caption{Oszilloskopfaufnahmen ohne und mit Verstärker.}
    \label{fig:oszilloskop}
  \end{figure}

Es ist erkennbar, dass die Messung mit dem Verstärker ein klareres, stärkeres Signal liefert.
\subsection{Strahlcharakteristik und Wirkungsquerschnitte}
Für die Folien werden die gemessenen Zählraten über \autoref{eq:I0} mit \autoref{eq:dichte} und \autoref{eq:omega}
in differentielle Streuquerschnitte umgerechnet.
Dafür wird für beide Folien die Folienstärke $d = \qty{2}{\micro\meter}$ und die Abmessung der Apparatur mit
$b_x = \qty{2}{\milli\meter}$, $b_y = \qty{10}{\milli\meter}$ und $l = \qty{45}{\milli\meter}$ verwendet.
Die einfallende Zählrate wird über die Aktivität des Präparats multipliziert mit dem Raumwinkel des Targets berechnet.
Mit der Aktivität von \SI{330}{\kilo\becquerel} im Jahr 1994, einer Halbwertszeit von $432$ Jahren und einer Blendenfläche von \SI{20}{\milli\meter\squared} ergibt sich eine
einfallende Zählrate von $C_0 = \qty{155.805}{\per\second}$.

Die Messwerte werden mit einem Fit der Form
\begin{equation}
    f(\theta) = \frac{a}{\sin^4\left(\frac{\theta}{2}\right)}
    \label{eq:sinfit}
\end{equation}
versehen und der theoretische Streuquerschnitt nach \autoref{eq:rutherford} dargestellt. Die dafür verwendeten Parameter lauten
\begin{align*}
    z &= 2 \\
    E_\text{\alpha} &= \qty{8.78954e-13}{\joule}
\end{align*}
\cite{241am}.
\subsubsection{Vakuum}
Die Strahlcharakteristik wird in einem Vakuum von \SI{0}{\milli\bar} durchgeführt, welches konstant aufrecht gehalten wird.
Die gemessenen Zählraten sind in \autoref{fig:vakuum} dargestellt.

\begin{figure}[H]
    \centering
    \includegraphics[width=\textwidth]{plots/meas_Vac.pdf}
    \caption{Strahlcharakteristik des Präparats im Vakuum. Ebenfalls dargestellt ist ein Gauß-Fit für die positiven Winkel, wobei Ausreißer nicht beachtet wurden.}
    \label{fig:vakuum}
\end{figure}

Die Zählraten folgen grob einer Normalverteilung um \SI{0}{\degree}. Zuerst wurde von \SI{0}{\degree} ausgehend die Zählrate für negative Winkel vermessen,
nach einer Fehleinstellung beim Wechsel auf positive Winkel musste die Apparatur neu geeicht werden.
Ebenfalls wurde nach dem ersten Ausreißer bei \SI{3}{\degree} der Diskriminator erhöht.

Die Parameter des Gaußfits
\begin{equation*}
    f(\theta) = a \cdot e^{-b \cdot x^2}
\end{equation*}

lauten

\begin{align*}
    a &= \qty{25.782(1268)}{\per\second} \\
    b &= \num{0.054(8)} \frac{1}{\qty{1}{\degree\squared}}.
\end{align*}

Es wird nur die positive Seite der Messung für den Fit berücksichtigt, da auf dieser Seite die nachfolgenden Messungen stattfinden und die erneute Eichung
in der Durchführung zu einer Unvergleichbarkeit der beiden Seiten führt.

Die Zählrate nimmt mit steigendem Winkel rapide ab, was auf fehlende Stöße zurückzuführen ist, die die \alpha-Teilchen aus ihrer Bahn lenken würden.

\subsubsection{Gold}
Für die Umrechnung der Zählraten in den Wirkungsquerschnitt nach \autoref{eq:I0} werden $\rho_\text{Au} = \qty{19320}{\kilo\gram\per\meter\cubed}$ und $M_\text{Au} = \qty{0.197}{\kilo\gram\per\mole}$ \cite{gold} verwendet.
Die Ordnungszahl ist $Z_\text{Au} = 79$. Die gemessenen Zählraten sind in \autoref{fig:gold} dargestellt.

\begin{figure}[H]
    \centering
    \includegraphics[width=\textwidth]{plots/dsdO_Au.pdf}
    \caption{Winkelabhängiger Wirkungsquerschnitt von einer \SI{2}{\micro\meter} dicken Goldfolie. Ebenfalls dargestellt ist ein Fit nach \autoref{eq:sinfit} und der theoretische Wirkungsquerschnitt nach \autoref{eq:rutherford}.}
    \label{fig:gold}
\end{figure}

Der Parameter des Fits nach \autoref{eq:sinfit} lautet
\begin{equation*}
    a_\text{Au} = \qty{7.627(138)e-28}{\meter\per\steradian\squared}.
\end{equation*}

Die Streuquerschnitte folgen dem Verlauf der Theoriekurve qualitativ, allerdings ist eine Verschiebung der Werte nach oben zu erkennen.
Werte bei höheren Winkeln liegen näher an der Theoriekurve als jene bei kleineren Winkeln. 

\subsubsection{Bismut}
Für die Umrechnung der Zählraten in den Wirkungsquerschnitt nach \autoref{eq:I0} werden $\rho_\text{Bi} = \qty{9790}{\kilo\gram\per\meter\cubed}$ und $M_\text{Bi} = \qty{0.209}{\kilo\gram\per\mole}$ \cite{bismut} verwendet.
Die Ordnungszahl ist $Z_\text{Bi} = 83$. Die gemessenen Zählraten sind in \autoref{fig:bismut} dargestellt.

\begin{figure}[H]
    \centering
    \includegraphics[width=\textwidth]{plots/dsdO_Bi.pdf}
    \caption{Winkelabhängiger Wirkungsquerschnitt von einer \SI{2}{\micro\meter} dicken Bismutfolie. Ebenfalls dargestellt ist ein Fit nach \autoref{eq:sinfit} und der theoretische Wirkungsquerschnitt nach \autoref{eq:rutherford}.}
    \label{fig:bismut}
\end{figure}

Der Parameter des Fits nach \autoref{eq:sinfit} lautet
\begin{equation*}
    a_\text{Bi} = \qty{2.513(167)e-28}{\meter\per\steradian\squared}.
\end{equation*}

Auch hier liegen die gemessenen Streuquerschnitte über der Theoriekurve, wobei die Abweichung bei kleineren Winkeln größer ist als bei größeren, im Falle von Bismut
ist dieser Unterschied bei den ersten Messwerten besonders stark ausgeprägt, wird dann aber für Winkel ab \SI{7}{\degree} beinahe irrelevant. 

