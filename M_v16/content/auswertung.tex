\section{Auswertung}
\label{sec:Auswertung}

Für die Auswertung wird die \texttt{Python}-Bibliothek \texttt{numpy} \cite{numpy} benutzt. Die Fits entstehen mit \texttt{curve\_fit} aus \texttt{scipy.optimize} \cite{scipy}.
Die Fehlerrechnung wird mit \texttt{uncertainties} \cite{uncertainties} durchgeführt. Plots entstehen mit \texttt{matplotlib.pyplot} \cite{matplotlib}.

Für die Folien werden die gemessenen Zählraten über \autoref{eq:I0} mit \autoref{eq:dichte} und \autoref{eq:omega}
in differentielle Streuquerschnitte umgerechnet.
Dafür wird für beide Folien die Folienstärke $d = \qty{2}{\micro\meter}$ und die Abmessung der Apparatur mit
$b_x = \qty{2}{\milli\meter}$, $b_y = \qty{10}{\milli\meter}$ und $l = \qty{45}{\milli\meter}$ verwendet.
Die einfallende Zählrate wird über die Aktivität des Präparats multipliziert mit dem Raumwinkel des Targets berechnet.
Mit der Aktivität von \SI{330}{\kilo\becquerel} im Jahr 1994, einer Halbwertszeit von $432$ Jahren und einer Blendenfläche von \SI{20}{\milli\meter\squared} ergibt sich eine
einfallende Zählraten von $C_0 = \qty{155.805}{\per\second}$.

Die Messwerte werden mit einem Fit der Form
\begin{equation}
    f(\theta) = \frac{a}{\sin^2\left(\frac{\theta}{2}\right)}
    \label{eq:sinfit}
\end{equation}
versehen und der theoretische Streuquerschnitt nach \autoref{eq:rutherford} dargestellt. Die dafür verwendeten Parameter lauten
\begin{align*}
    z &= 2
    E_\text{\alpha} &= \qty{8.78954e-13}{\joule}.
\end{align*}

\subsection{Vakuum}

Die Strahlcharakteristik wird in einem Vakuum von \SI{0}{\milli\bar} durchgeführt, welches konstant aufrecht gehalten wird.
Die gemessenen Zählraten befinden sich in \autoref{fig:vakuum}.

\begin{figure}[H]
    \centering
    \includegraphics[width=\textwidth]{plots/meas_Vac.pdf}
    \caption{Strahlcharakteristik des Präparats im Vakuum. Ebenfalls dargestellt ist ein Gauß-Fit für die positiven Winkel, wobei Ausreißer nicht beachtet wurden.}
    \label{fig:vakuum}
\end{figure}

Die Zählraten folgen grob einer Normalverteilung um \SI{0}{\degree}. Zuerst wurde von \SI{0}{\degree} ausgehend die Zählrate für negative Winkel vermessen,
nach einer Fehleinstellung beim Wechsel auf positive Winkel musste die Apparatur neu geeicht werden.
Ebenfalls wurde nach dem ersten Ausreißer bei \SI{3}{\degree} der Diskriminator erhöht.

Die Parameter des Gaußfits
\begin{equation*}
    f(\theta) = a \cdot e^{-b \cdot x^2}
\end{equation*}

lauten

\begin{align*}
    a &= \qty{25.782(1268)}{\per\second} \\
    b &= \num{0.054(8)} \frac{1}{\qty{1}{\degree\squared}}.
\end{align*}


\subsection{Gold}
Für die Umrechnung der Zählraten in den Wirkungsquerschnitt nach \autoref{eq:I0} werden $\rho_\text{Au} = \qty{19320}{\kilo\gram\per\meter\cubed}$ und $M_\text{Au} = \qty{0.197}{\kilo\gram\per\mole}$ verwendet.
Die Ordnungszahl ist $Z_\text{Au} = 79$.



\subsection{Bismut}