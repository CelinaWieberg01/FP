\section{Diskussion}
\label{sec:Diskussion}

Die Auswertung ist insbesondere dadurch eingeschränkt, dass die Peaks händisch gesucht werden mussten. Aus diesem Grund sind sehr kleine Peaks vernachlässigt
und nur eindeutig erkennbare Energien mit genügend großen Wahrscheinlichkeiten berücksichtig worden. Der Einfluss dieser Tatsache ist aber gering und würde lediglich Mittelwerten und Fits kleinere Unsicherheiten geben.

\subsection{Europium-152}

Die Kalibrierungsfunktion \eqref{eq:kanalenergie}
\begin{equation}
    E(K) = (0,103304 \pm 0,000044) \si{\kilo\electronvolt} \, \cdot K - (0,903881 \pm 0,188430) \, \si{\kilo\electronvolt}
\end{equation}

weist in ihrer Steigung eine sehr kleine Unsicherheit auf, der $y$-Achsenabschnitt ist jedoch mit ungefähr $20 \%$ vergleichsweise hoch.
Zu erwarten wäre ein $y$-Achsenabschnitt von $0$, da der Kanal $0$ keine Energie beschreiben würde. Alternativ wäre ein positiver Achsenabschnitt
für die Berücksichtigung von Untergrund denkbar. Da der Detektor allerdings eine untere Nachweisgrenze von $\qty{50}{\kilo\electronvolt}$ hat, spielt dieser Einfluss keine Rolle.\\

Die berechnete Aktivität von $A = \qty{1202(17)}{\becquerel}$ ist im Hinblick auf die Ursprungsaktivität plausibel.\\

Die berechneten Vollenergienachweiswahrscheinlichkeiten in \ref{tab:europiumeffizienz} sind im Vergleich zu Altprotokollen sehr klein, diese liegen meist in der Größenordnung von $\num{1e-2}$ bis $\num{1e-1}$. Entsprechend sind die Parameter
der Fitfunktion
\begin{equation}
    Q(E) = \num{0.3290(307)} (\frac{E}{\si{\kilo\electronvolt}})^{\num{-0.8472(169)}}.
\end{equation}
sehr klein. Das Verhalten der Nachweiswahrscheinlichkeit entspricht der Erwartung, da niederenergetische Photonen unwahrscheinlicher sind, den Detektor ohne Wechselwirkungen zu durchdringen.\\

\subsection{Cäsium-137}

Für die Energien und Inhalte der signifikanten Stellen findet sich
\begin{table}[H]
    \centering
    \caption{Photopeak, Comptonkante, Rückstreupeak}
    \label{tab:fit}
    \begin{tabular}{c c c c}
        \toprule
        {Name} & {Energie ($\si{\kilo\electronvolt}$)} & Theoretische Energie ($\si{\kilo\electronvolt}$) & Inhalt \\
        \midrule
        {Photopeak} & 661,69 & 661,65 & \num{1,176(11)e4} \\
        {Comptonkante} & \num{457(7)} & 477 & 12125 \\
        {Rückstreupeak} & \num{191(4)} & 184 & {-} \\
        \bottomrule
    \end{tabular}
\end{table}

Die Abweichungen der Photopeaks ist weit unter $0,1 \%$. Der Unterschied der Comptonkante beträgt ungefähr $4 \%$, was in Anbetracht der Auflösung und des relativen Fehlers akzeptabel ist.
Die Abweichung im Rückstreupeak beträgt ungefähr $3 \%$. 

Die Inhalte der Photopeaks entsprechen nicht den Erwartungen in Anbetracht der Absorptionswahrscheinlichkeiten. Die Wahrscheinlichkeit für die Photoabsorption und für die Comptonstreuung,
\begin{align}
    P_\text{Photo} &= 2,693 \% \\
    P_\text{Compton} &= 76,37 \%
\end{align}
deuten auf einen ungefähr 25 mal größeren Linieninhalt des Comptonkontinuums an, da dieser 25 mal wahrscheinlicher ist.
Das Verhältnis von $\frac{\text{Linieninhalt(Compton)}}{\text{Linieninhalt(Photo)}} \approx 1,03$.


Die Gaußfunktion mit den Parametern
\begin{align}
    a &= \num{12233.183(110637)} \\
    \mu &= \num{6415.685(89)} \\
    \sigma &= \num{9.764(63)},
\end{align}
die eine Fehler von maximal ein paar Prozenten aufweisen, beschreiben den Peak augenscheinlich sehr gut. \\

Das Verhältnis von Zehntel- zu Halbwertsbreite,
\begin{equation}
    \frac{41.841}{22,922} \approx 1.825,
\end{equation}
ist akzeptabel, da es nur um ungefähr $0,1 \%$ vom Soll-Wert von $1,823$ abweicht.

\subsection{Barium-133}

Die mittlere Aktivität der Barium-Probe hat mit
\begin{equation}
    \bar{A}_\text{Ba} = \qty{801(10)}{\becquerel}
\end{equation}
eine kleine Unsicherheit, da fünf Peaks zur Bestimmung genutzt werden konnten.\\

\subsection{Unbekanntes Erz}

Insbesondere hier kommt die Schwierigkeit der Peaksuche zum Tragen. Viele mögliche Peaks mussten
wegen ihrer Uneindeutigkeit oder geringen Ausprägung vernachlässigt werden.\\

Eine mittlere Aktivitätsbestimmung der verschiedenen Isotope wird für zwecklos gehalten, da dasselbe Isotop, z.B. Blei, mit 
einer Aktivität von $\qty{3289(70)}{\becquerel}$ in einem Kanal und $\qty{6710(323)}{\becquerel}$ in einem anderen Kanal einen zu großen Unterschied aufweist.\\

