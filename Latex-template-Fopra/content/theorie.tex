%\documentclass{article}

\section{Zielsetzung}
\label{sec:Zielsetzung}
Ziel des Versuches ist es, die Grundlagen der Vakuumphysik, so wie den Umgang mit
den entsprechenden Vakuumtechnik-Komponenten zu erlernen. Dazu werden Evakuierungskurven und effektives Saugvermögen 
von Drehschieber- und Turbomolekularpumpen analysiert und ihre Leckraten bestimmt.


\section{Theorie}
\label{sec:Theorie}

Das Vakuum ist ein Zustand geringer Gasdichte, also ein Raum welcher Nahezu leer von Materie ist.
Da dieser Raum nur wenig Teilchen enthält, ist der Druck in einem Vakuum deutlich geringer als der Atmosphärendruck.
Druck ist definiert als die Kraft pro Fläche die von Gasmolekülen ausgeübt wird, wenn sie auf
eine Oberfläche stoßen.Die mittlere freie Weglänge, also die Strecke welche ein Teilchen im Mittel zurücklegt bevor es
mit einem anderen kollidiert ist folglich sehr hoch. Ein perfektes Vakuum ist in der Praxis nicht zu realisieren, zur mathematischen 
beschreibung des Vakuums verwendet man die Zustandsgleichungen des idealen Gases, einem theoretischen Modell für ein 
Gas in welchem die Teilchen keine Wechselwirkungen außer elastischen Stößen erfahren. Außerdem ist ihr Volumen Vernachlässigbar 
und sie werden als Punktförmige Teilchen angenommen.
Die Thermische Zustandsgleichung des idealen Gases ist gegeben durch:
\begin{equation}
     p \cdot V=NK_bT 
\end{equation}  
Dabei ist \( p \) der Druck, \( V \) das Volumen, \( N \) die Teilchenanzahl, \( k_B \) die Boltzmann-Konstante
 und \( T \) die Temperatur.
Bei konstanter Temperatur ist der Druck eines Gases umgekehrt proportional zu zum Volumen. 
Man definiert das Boyle-Mariottesche Gesetz:
\begin{equation}
    \frac{p_1}{p_2}=  \frac{V_2}{V_1} 
\end{equation}
Beim Evakuierungsvorgang in ein Vakuum sinkt der Druck über die Zeit, der normale Atmosphärendruck beträgt auf der Erdoberfläche
ein bar. Die Druckbereiche des Vakuums liegen bei folgenden Werten :
Grobvakuum bei 1 bar bis $10^-3$ bar, Feinvakuum von $10^-3$ bar bis $10^-7$ bar, Hochvakuum von $10^-7$ bar bis $10^-9$ bar und
das Hochvakuum liegt bei Werten unter $10^-9$ bar.
Bei niedrigen Druckbereich dominiert die Molekulare Strömung, hier bewegen sich die Moleküle fast unabhängig voneinander,
da sie nur selten kollidieren und ihre mittlere freie Weglänge groß ist. Bei größeren Druckbereichen kollidieren die Teilchen oft 
und bewegen sich in geordneten Bahnen, dass nennt man Laminare Ströumg, im Vakuum tritt diese allerdings praktisch nicht auf.
Im folgenden betrachten wir ein Gemisch aus Gasen, der Gesamtdruck ist dabei die Summe aller Partialdrücke. Also aller
Drücke aller Komponenten.




\cite{sample}
