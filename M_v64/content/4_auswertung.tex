\section{Analysis}
\label{sec:Auswertung}

For the analysis, the \texttt{Python} library \texttt{numpy} \cite{numpy} is used. Fits are done using \texttt{curve\_fit} from \texttt{scipy.optimize} \cite{scipy}.
Error analysis is done using \texttt{uncertainties} \cite{uncertainties}. Plots are created using \texttt{matplotlib.pyplot} \cite{matplotlib}.

\subsection{Contrast}

The exact measurements for maximum and minimum intensities depending on the angle of polarization are given in \autoref{tab:contrast}. Due to large fluctuations in the measurement for high voltages, the error of a data point is set at \SI{10}{\percent} of the measured voltage.
Using \autoref{eq:contrast}, the contrast $C$ can be calculated.

Additionally, the angular dependence of the contrast is described by \autoref{eq:contrast_phi}.
Considering possible sources of errors, one model can be chosen to be
\begin{equation*}
    C_{\mathrm{model}}(\phi)
    = C_{\mathrm{off}} + C_0 \, \left| \sin(2\phi) \right| ,
    \label{eq:contrast_model}
\end{equation*}
where $C_0$ denotes the modulation amplitude of the interferometer and 
$C_{\mathrm{off}}$ accounts for a residual background contrast caused by optical 
imperfections or slight misalignments in the setup.
The results are shown in \autoref{fig:contrast}.

\begin{figure}[H]
    \includegraphics[width=0.9\textwidth]{plots/contrast.pdf}
    \caption{Measured contrast $C(\phi)$ as a function of the polarization angle, together with the fitted model.}
    \label{fig:contrast}
\end{figure}

The parameters $C_0$ and $C_{\mathrm{off}}$ were obtained from a weighted 
least-squares fit, where each data point was weighted by $1/\Delta C(\phi_i)^2$. 
The uncertainties of the fitted parameters correspond to the square roots of the 
diagonal elements of the covariance matrix.

The fit yields the following parameter values:
\begin{align*}
    C_0 &= \num{0.8848(55)} , \\
    C_{\mathrm{off}} &= \num{0.0199(48)} .
\end{align*}

All subsequent measurements in this experiment were performed at a fixed polarization 
angle of $\phi = 135^\circ$ where the contrast is close to its maximum. 
This choice ensures a high sensitivity of the interferometer to small phase changes 
in the following measurements.


\subsection{Refractive index of glass}

In the experimental configuration,
the vacuum wavelength of the laser is 
\[
\lambda_{\mathrm{vac}} = \SI{632.99}{\nano\meter},
\]

the initial rotation angle of the glass plates is 

\[
\Theta_0 = 10^\circ,
\]

and the thickness of each glass plate is 

\[
d = 1\,\mathrm{mm}.
\]


Due to the high sensitivity of the counting device, counting uncertainty of two fringes was assumed for $M$, and the angular precision
of the rotation stage was taken as $\Delta \Theta \approx \SI{2}{\degree}$.

Using these uncertainties, the refractive index of glass $n$ was calculated for each measurement using \autoref{eq:glass}, 
\begin{equation*}
    n_{\mathrm{glass}} = \num{1.4799(1429)}.
\end{equation*}

The deviation from the literature value of \num{1.5151} is \SI{2.32}{\percent}.


\subsection{Refractive index of air}

The measurements for the pressure dependent count of fringes as well as the corresponding refractive index differences are listed in \autoref{tab:air}.
The cell had a length of $L = (100.0 \pm 0.1)\,\mathrm{mm}$.
Using \autoref{eq:n_M}, the difference in refractive index from vacuum to air can be calculated from the counted number of fringes at a given pressure.
The results for all measurements are shown in \autoref{fig:n_M}. A linear function 

\begin{equation*}
    \Delta n = m \cdot p + b
\end{equation*}

with $m$ in $\si{\per\pascal}$ can be used to fit the data, also shown in \autoref{fig:n_M}
\begin{figure}[H]
    \centering
    \includegraphics[width=\textwidth]{plots/air.pdf}
    \caption{The calculated difference in the refractive index between a vacuum and air of all three measurements, including a linear fit.}
    \label{fig:n_M}
\end{figure}

The fit parameters are determined to be
\begin{align*}
    m &= \qty{2.849(78)e-9}{\per\pascal} \\
    b &= \num{2.605(490)e-5} 
\end{align*}

Using \autoref{eq:lorentzlorenz} with the universal gas constant $R = 8.3144\,\mathrm{J/(mol\,K)}$,
the molar refractivity of air $A$ for this experiment can be determined from the linear fit,

\begin{equation*}
    m = \frac{3}{2}\frac{A}{RT}
\end{equation*}

with the previously calculated slope $m$ and the measured room temperature $T = \SI{22}{\celsius}$.
It follows that
\begin{equation*}
    A = \SI{4.68(13)e-6}{\meter\cubed\per\mole}.
\end{equation*}
Inserting this result into the Lorentz-Lorenz law again, the refractive index of air at standard room temperature $T_0 = \SI{15}{\celsius}$
and standard room pressure $p_0 = \SI{1013}{\hecto\pascal}$ can be calculated,
\begin{equation*}
    n_\text{air, standard} = \num{1.000323(9)},
\end{equation*}
or more accurately, the difference in refractive index is
\begin{equation*}
    \Delta n_\text{air, standard} = \num{0.000323(9)}.
\end{equation*}

The deviation from the literature value of \num{0.000276} is \SI{17.0}{\percent}.