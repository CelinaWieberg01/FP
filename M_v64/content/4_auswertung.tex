\section{Analysis}
\label{sec:Auswertung}

For the analysis, the \texttt{Python} library \texttt{numpy} \cite{numpy} is used. Fits are done using \texttt{curve\_fit} from \texttt{scipy.optimize} \cite{scipy}.
Error analysis is done using \texttt{uncertainties} \cite{uncertainties}. Plots are created using \texttt{matplotlib.pyplot} \cite{matplotlib}.

\subsection{Determination of the Contrast as a Function of the Polarization Angle}

For each selected polarization angle~$\phi$, three measurements of the maximum and 
minimum intensities, $I_{\mathrm{max}}(\phi)$ and $I_{\mathrm{min}}(\phi)$, were recorded. 
From these values, the arithmetic mean and the standard deviation were calculated. 
Using Gaussian error propagation, the uncertainties of the derived quantities were 
determined consistently.

The contrast is defined as
\begin{equation}
    V(\phi) =
    \frac{I_{\mathrm{max}}(\phi) - I_{\mathrm{min}}(\phi)}
         {I_{\mathrm{max}}(\phi) + I_{\mathrm{min}}(\phi)} ,
    \label{eq:contrast_definition}
\end{equation}
and the corresponding uncertainty $\Delta V(\phi)$ follows from the uncertainties of 
$I_{\mathrm{max}}$ and $I_{\mathrm{min}}$. 
The resulting contrast values, including their error bars, are shown in 
Fig.~\ref{fig:contrast_plot}.

\begin{figure}[h!]
    \includegraphics[width=0.9\textwidth]{M_v64/plots/contrast.png}
    \caption{Measured contrast $V(\phi)$ as a function of the polarization angle, 
    together with the fitted model.}
    \label{fig:contrast_plot}
\end{figure}

The angular dependence of the contrast is described by the model
\begin{equation}
    V_{\mathrm{model}}(\phi)
    = V_{\mathrm{off}} + V_0 \, \left| \sin(2\phi) \right| ,
    \label{eq:contrast_model}
\end{equation}
where $V_0$ denotes the modulation amplitude of the interferometer and 
$V_{\mathrm{off}}$ accounts for a residual background contrast caused by optical 
imperfections or slight misalignments in the setup.

The parameters $V_0$ and $V_{\mathrm{off}}$ were obtained from a weighted 
least-squares fit, where each data point was weighted by $1/\Delta V(\phi_i)^2$. 
The uncertainties of the fitted parameters correspond to the square roots of the 
diagonal elements of the covariance matrix.

The fit yields the following parameter values:
\begin{align}
    V_0 &= 0.8848 \pm 0.0095 , \\
    V_{\mathrm{off}} &= 0.0199 \pm 0.0083 .
\end{align}

The fitted curve is shown together with the experimental data in 
Fig.~\ref{fig:contrast_plot}. 
All subsequent measurements in this experiment were performed at a fixed polarization 
angle of $\phi = 135^\circ$, where the contrast is close to its maximum. 
This choice ensures a high sensitivity of the interferometer to small phase changes 
in the following measurement steps.

\subsection{Refractive Index of Glass}

To determine the refractive index of glass, a pair of parallel glass plates was inserted
into the two interferometer arms. Rotating the glass plates introduces an additional
optical path difference between the beams, which leads to a measurable number of
interference fringes passing through the detector. In the experimental configuration,
the vacuum wavelength of the laser is 


\[
\lambda_{\mathrm{vac}} = 632.99\,\mathrm{nm},
\]


the initial rotation angle of the glass plates is 


\[
\Theta_0 = 10^\circ,
\]


and the thickness of each glass plate is 


\[
d = 1\,\mathrm{mm}.
\]



By combining Eq.~(10) and Eq.~(12), the following relation between the measured
number of fringes $M$ and the refractive index $n$ of the glass is obtained:
\begin{equation}
    n = 
    \frac{1}{
    1 - 
    \dfrac{\lambda_{\mathrm{vac}}\, M}{
    2 d\, \Theta_0\, \Theta}},
    \label{eq:glass_index_formula}
\end{equation}
where $\Theta$ denotes the rotation angle of the glass plates during the measurement.

For each rotation step, the number of passing fringes was recorded.
A counting uncertainty of one fringe was assumed for $M$, and the angular precision
of the rotation stage was taken as 


\[
\Delta \Theta \approx 1^\circ.
\]


Using these uncertainties, the refractive index was calculated for each measurement,
and the corresponding errors were obtained via Gaussian error propagation.

From the full dataset, the following statistical quantities were determined:
\begin{align}
    n_{\mathrm{mean}} &= 1.4799 , \\
    n_{\mathrm{std}}  &= 0.0682 , \\
    \Delta n_{\mathrm{mean}} &= 0.0220 .
\end{align}

The final result for the refractive index of the glass sample is therefore
\begin{equation}
    n_{\mathrm{glass}} = 1.4799 \pm 0.0220 .
\end{equation}


\subsection{Refractive Index of Air}

The refractive index of air depends on its density and therefore varies with pressure.
To quantify this dependence, the number of interference fringes passing through the
centre of the interference pattern was recorded for each pressure value between
50\,mbar and 996\,mbar. For every pressure point, three measurements 
$M_1(p)$, $M_2(p)$, and $M_3(p)$ were taken and subsequently averaged. 
The corresponding data are listed in Table~3.


A change in the refractive index of the air cell leads to a change in the optical path
length, which manifests as a shift of the interference fringes.  
Combining Eq.~(10) and Eq.~(13) yields the relation
\begin{equation}
    \Delta n(p) 
    = \frac{M(p)\,\lambda_{\mathrm{vac}}}{L}
    = n(p) - 1 ,
    \label{eq:deltan_definition}
\end{equation}
where $\lambda_{\mathrm{vac}} = 632.996\,\mathrm{nm}$ is the laser wavelength and 
$L = (100.0 \pm 0.1)\,\mathrm{mm}$ is the length of the air cell.  
The calculated values of $\Delta n$ are listed in Table~3.


According to Eq.~(15), the refractive index of an ideal gas at pressure $p$ and temperature $T$
is expected to follow the linear relation
\begin{equation}
    \Delta n(p, T) 
    = A \, \frac{3p}{2RT} ,
    \label{eq:deltan_theory}
\end{equation}
where $R = 8.3144\,\mathrm{J/(mol\,K)}$ is the universal gas constant and $A$ is the molar
refractivity of air.  
Thus, $\Delta n$ should increase linearly with pressure.

Figure~\ref{fig:air_plot} shows the experimentally determined values of $\Delta n(p)$ together
with a linear fit of the form
\begin{equation}
    f(p) = c\,p + b .
    \label{eq:linear_fit}
\end{equation}

\begin{figure}[h!]
    \centering
    \includegraphics[width=0.9\textwidth]{plots/air.png}
    \caption{Measured values of $\Delta n$ as a function of pressure, together with a
    linear fit according to Eq.~\eqref{eq:linear_fit}.}
    \label{fig:air_plot}
\end{figure}

\subsection*{Fit results and determination of the molar refractivity}

From the weighted least-squares fit, the parameters were obtained as
\begin{align}
    c &= (2.560 \pm 0.018)\times 10^{-9}\,\mathrm{Pa^{-1}}, \\
    b &= (1.682 \pm 0.120)\times 10^{-5}.
\end{align}

The slope $c$ is directly related to the molar refractivity $A$ via
\begin{equation}
    c = A\,\frac{3}{2RT}.
\end{equation}
Solving for $A$ and inserting the measurement temperature 
$T = 22.0^\circ\mathrm{C}$ yields
\begin{equation}
    A = (4.191 \pm 0.030)\times 10^{-6}\,\mathrm{m^3/mol}.
\end{equation}


Using the obtained value of $A$, the refractive index of air at arbitrary temperature
$T_0$ and pressure $p_0$ follows from
\begin{equation}
    n_{\mathrm{air}}(T_0, p_0)
    = 1 + A\,\frac{3p_0}{2RT_0} + b .
\end{equation}

For the reference conditions $T_0 = 15.0^\circ\mathrm{C}$ and 
$p_0 = 1013.0\,\mathrm{hPa}$, this yields
\begin{align}
    \Delta n_{\mathrm{air}}(T_0, p_0) &= (2.835 \pm 0.022)\times 10^{-4}, \\
    n_{\mathrm{air}}(T_0, p_0) &= 1.0002826 \pm 0.0000022 .
\end{align}

Here, $\Delta n = n - 1$ denotes the deviation of the refractive index from that of
vacuum ($n_0 = 1$).  
The obtained value agrees well with literature values for the refractive index of air
under similar environmental conditions.
