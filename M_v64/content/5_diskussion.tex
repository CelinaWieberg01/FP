\section{Discussion}
\label{sec:Diskussion}

The interferometric measurements performed in this experiment provide a coherent and
mutually consistent picture of the optical properties of the interferometer and the
materials placed in its beam paths. By analysing three different aspects of the setup
the contrast as a function of the polarization angle, the refractive index of glass,
and the pressure dependence of the refractive index of air it is possible to assess
both the performance of the interferometer and the reliability of the extracted physical
quantities.

The contrast measurement demonstrates that the interferometer operates with high fringe
visibility when the polarization angle is chosen appropriately. The fitted modulation
amplitude of $V_0 = 0.8848 \pm 0.0095$ indicates that the interferometer is capable of
producing nearly ideal interference fringes. The small residual background contrast
$V_{\mathrm{off}} = 0.0199 \pm 0.0083$ is physically reasonable and can be attributed to
imperfect extinction in the polarizing elements, residual stray light, or slight
misalignments in the optical components. The excellent agreement between the measured
contrast values and the model $V_{\mathrm{model}}(\phi) = V_{\mathrm{off}} + V_0
|\sin(2\phi)|$ confirms that the polarization dependence of the interferometer is well
understood. Performing all subsequent measurements at $\phi = 135^\circ$, where the
contrast is close to its maximum, is therefore well justified and ensures optimal
sensitivity to phase variations.

The determination of the refractive index of glass yields a value of


\[
n_{\mathrm{glass}} = 1.4799 \pm 0.0220.
\]


This result is in very good agreement with typical literature values for common optical
glasses at visible wavelengths, such as BK7 or soda-lime glass, which generally lie in
the range $n \approx 1.47$--$1.52$ at $\lambda \approx 633\,\mathrm{nm}$. The relative
uncertainty of approximately $1.5\%$ is reasonable for a manual interferometric
measurement in which the dominant sources of error are the finite angular resolution of
the rotation stage and the assumed counting uncertainty of one fringe. Small systematic
effects, such as deviations from the nominal plate thickness or slight wedge angles,
may also contribute. Nevertheless, the measured value lies comfortably within the
expected range, demonstrating that the interferometer is capable of reliably determining
refractive indices of solid materials.

The pressure-dependent measurement of the refractive index of air probes a much smaller
effect, since $\Delta n = n - 1$ for air is on the order of $10^{-4}$. Despite this, the
data exhibit a clear linear dependence on pressure, as predicted by the ideal gas
relation. The linear fit yields


\[
c = (2.560 \pm 0.018)\times 10^{-9}\,\mathrm{Pa^{-1}}, \qquad
b = (1.682 \pm 0.120)\times 10^{-5},
\]


from which the molar refractivity is obtained as


\[
A = (4.191 \pm 0.030)\times 10^{-6}\,\mathrm{m^3/mol}.
\]


This value is consistent with standard literature values for dry air in the visible
range. Using the fitted parameters, the refractive index of air at standard atmospheric
conditions ($T_0 = 15.0^\circ\mathrm{C}$, $p_0 = 1013\,\mathrm{hPa}$) is found to be


\[
n_{\mathrm{air}}(T_0,p_0) = 1.0002826 \pm 0.0000022,
\]


which agrees very well with typical tabulated values of $n_{\mathrm{air}} \approx
1.00027$--$1.00029$ at $\lambda \approx 633\,\mathrm{nm}$. The small absolute
uncertainty demonstrates the high sensitivity of the interferometric method, even when
probing refractive index changes several orders of magnitude smaller than those
observed in glass.

Taken together, the three measurements illustrate both the versatility and the
precision of interferometric techniques. The contrast analysis confirms that the
interferometer is well aligned and capable of producing high-visibility fringes. The
glass measurement shows that refractive indices of solid materials can be determined
with percent-level accuracy. The air measurement, in turn, highlights the ability of
the interferometer to detect extremely small changes in optical path length, yielding a
refractive index in excellent agreement with literature values. The main sources of
systematic uncertainty across all measurements are the manual fringe counting, the
finite angular resolution of the rotation stage, and possible calibration uncertainties
in pressure and temperature. Nevertheless, within the scope of this experiment, the
results are highly consistent and demonstrate the reliability of the interferometric
approach.
